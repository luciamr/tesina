\section{Clustering}

El \textit{Clustering} se encarga de dividir conjuntos de objetos en grupos (\textit{clusters}) significativos
o útiles. Para llevar a cabo esta tarea debe basarse solamente en la información que describe
a los objetos y sus relaciones. En algunos casos el grado de semejanza se basa en la estrucutura del objeto, en
otros, en características o propiedades del mismo. El objetivo es que todos los objetos que se encuentran en un
mismo \textit{cluster} sean similares o estén relacionados entre sí y sean, a su vez, diferentes de los objetos
que se encuentran en otros \textit{clusters} o no estén vinculados a éstos. A mayor similitud dentro de cada grupo
y mayor diferencia entre grupos, mejor y más claro es el \textit{clustering}. 

El \{Clustering} puede ser visto como una forma de clasificación ya que etiqueta los objetos de acuerdo a la
clase (\textit{cluster}) a la que pertenecen. Al basar este etiquetado exclusivamente en la información de los
objetos que está clasificando y no en un modelo desarrollado previamente, se lo puede considerar como una forma
de clasificación no supervisada.

Al conjunto completo de \textit{clusters} de una collección se lo llama \textit{clustering}. Existen distintos
tipos de \textit{clustering} de acuerdo a sus características:

\begin{itemize}
 \item Particional/Jeráquico
 Un \textit{clustering} particional divide un cojunto de objetos en subconjuntos disjuntos, de manera que cada
 objeto pertenece a solamente un subconjunto. En cambio, un \textit{clustering} jerárquico propone \textit{clusters}
 anidados organizados en forma de árbol, cada nodo del árbol es la unión de sus hijos, siendo la raíz del árbol
 el \textit{cluster} que contiene a todos los objetos.
 \item Exclusivo/Superpuesto/Difuso
 En un \textit{clustering} exclusivo cada objeto pertenece a un único \textit{cluster}, mientras que los
 superpuestos o no exclusivos se utilizan para poder indicar que un mismo objeto puede pertenecer simultáneamente
 a más de una clase.
 
\end{itemize}
