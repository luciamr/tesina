\section{Clustering}

El \textit{Clustering} se encarga de dividir conjuntos de objetos en grupos (\textit{clusters}) significativos
o útiles. Para llevar a cabo esta tarea debe basarse solamente en la información que describe
a los objetos y sus relaciones. En algunos casos el grado de semejanza se basa en la estrucutura del objeto, en
otros, en características o propiedades del mismo. El objetivo es que todos los objetos que se encuentran en un
mismo \textit{cluster} sean similares o estén relacionados entre sí y sean, a su vez, diferentes de los objetos
que se encuentran en otros \textit{clusters} o no estén vinculados a éstos. A mayor similitud dentro de cada grupo
y mayor diferencia entre grupos, mejor y más claro es el \textit{clustering} \parencite{pang2006introduction}. 

El \textit{Clustering} puede ser visto como una forma de clasificación ya que etiqueta los objetos de acuerdo a la
clase (\textit{cluster}) a la que pertenecen. Al basar este etiquetado exclusivamente en la información de los
objetos que está clasificando y no en un modelo desarrollado previamente, se lo puede considerar como una forma
de clasificación no supervisada.

Al conjunto completo de \textit{clusters} de una collección se lo llama \textit{clustering}. Existen distintos
tipos de \textit{clustering} de acuerdo a sus características:

\begin{itemize}
 \item Particional/Jeráquico \\
 Un \textit{clustering} particional divide un cojunto de objetos en subconjuntos disjuntos, de manera que cada
 objeto pertenece a solamente un subconjunto. En cambio, un \textit{clustering} jerárquico propone \textit{clusters}
 anidados organizados en forma de árbol, cada nodo del árbol es la unión de sus hijos, siendo la raíz del árbol
 el \textit{cluster} que contiene a todos los objetos.
 \item Exclusivo/Superpuesto/Difuso \\
 En un \textit{clustering} exclusivo cada objeto pertenece a un único \textit{cluster}, mientras que los
 superpuestos o no exclusivos se utilizan para poder indicar que un mismo objeto puede pertenecer simultáneamente
 a más de una clase. En cambio, en un \textit{clustering} difuso, todos los objetos pertenecen a todos los \textit{clusters}
 con distintos grados de pertenencia, la cual varía entre 0 (enteramente no pertenece) y 1 (pertenece absolutamente),
 es decir, los \textit{clusters} son manejados como conjuntos difusos.
 \item Completo/Parcial \\
 En un \textit{clustering} completo cada objeto es asignado a un \textit{cluster}, mientras que en uno parcial
 esto no sucede, algunos objetos pueden no pertenecer a ninguno.
\end{itemize}

Los \textit{clusters} propiamente dichos pueden clasificarse de la siguiente manera:

\begin{itemize}
 \item Separados \\
 Cada objeto está más cerca de los demás objetos del \textit{cluster} que de cualquier otro objeto que no pertenzca
 al mismo grupo. Algunas veces se utiliza un umbral para asegurar que todos los  objetos de un mismo \textit{cluster}
 estén lo suficientemente cerca entre sí. Ésto puede satifacerse sólo cuando la información contiene \textit{clusters}
 naturales que se encuentran bastante alejados unos de otros.
 \item Basados en Prototipos \\
 Cada objeto está más cerca del prototipo que define el \textit{cluster} que de cualquier otro prototipo de otro
 \textit{cluster}. Cuando los atributos de los objetos son continuos, el prototipo suele ser un centroide (la media de
 los puntos del \textit{cluster}, mientras que cuando son categóricos, habitualmente el prototipo es un menoide
 (el punto más representativo del \textit{cluster}). Mientras que un centroide casi nunca corresponde a un objeto, un
 menoide, por definición, debe serlo. En general, este tipo de \textit{clusters} tiende a ser globular.
 \item Basados en Grafos \\
 Si la información se representa como un grafo, donde cada nodo es un objeto y los arcos son conexiones entre los
 objetos, un \textit{cluster} puede verse como una componente conectada, un conjunto de objetos que están conectados
 entre sí, pero no tienen ninguna conexión con objetos externos al grupo.
 \item Basados en Densidad \\
 Un \textit{cluster} es una región densa de objetos que se encuentra rodeada por una región de baja densidad.
 \item Propiedad compartida \\
 Cada \textit{cluster} es un conjunto de objetos que comparte una propiedad, ésto abarca las definiciones anteriores
 de \textit{cluster} e incorpora nuevas. Para poder llevarlo a cabo es necesario un algoritmo de \textit{clustering}
 con un concepto de \textit{cluster}lo suficientemente específico como para poder detectar exitosamente los
 \textit{clusters}. El proceso de encontrar dichos \textit{clusters} recibe el nombre de \textit{clustering} conceptual.
\end{itemize}

Dos de las técnicas de \textit{Clustering} basados en prototipos más importantes son \textit{K-means} y \textit{K-medoid}.
Por un lado, \textit{K-means} define un prototipo en función de un centroide y en general se aplica a objetos en
espacios n-dimensionales contínuos. Por el otro, \textit{K-medoid} define en función de un menoide, haciendo que se
pueda aplicar a diferentes tipos de información, ya que sólo requiere una medida de proximidad entre dos objetos.
