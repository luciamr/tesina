\section{Fundamentos y estado del conocimiento sobre el tema}
\iffalse
Escriba una breve introducción general al tema y cite y comente las mayores
contribuciones en el tema específico, incluyendo bibliografıa actualizada.
\fi
La clasificación de imágenes o videos en categorías semánticas es un problema de interés tanto para la comunidad científica como
para la industria. La detección de diferentes tipos de escenas por lo general se basa en vectores de características que describen
el color y la textura de las imágenes entre otras propiedades visuales.

Hace ya más de una década, comenzó a verse una tendencia a usar \textit{keypoints} y puntos de interés local en la recuperación y
clasificación de la información contendida en la imágenes. ***** Los \textit{keypoints} son zonas destacadas de las imágenes, que contienen
abundante información local acerca de la misma, los cuales pueden ser identificados usando diferentes detectores *** y respresentados
por diversos descriptores.***

Una vez que los \textit{keypoints} son obtenidos, éstos son distribuidos en una gran cantidad de \textit{clusters}, asignando a un mismo
\textit{cluster} aquellos de características similares. Cada \textit{cluster} es considerado una \textit{visual word}
que representa el patrón local específico compartido por todos los \textit{keypoints} de ese \textit{cluster}, ésto permite obtener un
vocabulario de \textit{visual words} que describe todos los patrones locales de las imágenes. A partir
de ésto, una imagen puede respresentarse como una \textit{Bag of Visual Words}, un vector que contiene
la cantidad de veces que cada \textit{visual word} aparece en la imagen, el cual es usado como vector
de características durante la clasificación.

Ésta representación es análoga a la de Bag of Words utilizada en textos para describir tanto la forma
como la semántica. Ésto permite que muchas de las técnicas ya desarrolladas y usadas para el análisis de textos hayan podido ser aplicadas
al trabajo con imágenes.

Los \textit{keypoints} suelen encontrarse en los ángulos y bordes de los objetos presentes en las imágenes.
Uno de los métodos más utilizados para la detección de los \textit{keypoints} es la Diferencia Gaussiana (DoG), sin embargo existen otros como
Harris corner detector, Fast Hessian detector, AGAST y multi-scale AGAST. 

En cuanto a los descriptores, a lo largo de los años se han publicado numerosos desarrollos. Para trabajar con imágenes, alguno de los más utilizados
son SIFT\parencite{lowe2004distinctive} y FREAK\parencite{alahi2012freak}. Por lo general, cada descriptor se utiliza en conjunto a un detector
específico. En lo referente a videos, suelen utilizarse descriptores que se basan en los de imágenes e incorporan el factor temporal,
como MoSIFT\parencite{chen2009mosift} y MoFREAK\parencite{whiten2013mofreak}.

Una vez que se completa el proceso de obtención de información de los objetos a clasificar y construimos nuestro \textit{Bag of Visual
Words}, se llega a la etapa de clasificación propiamente dicha. Para completar esta última etapa uno de los métodos más utilizado
actualmente es \textit{Support Vector Machine} (SVM).

Se han hecho diversas pruebas en videos utilizando la idea de \textit{Bag of Visual Words}, algunas con
datasets de videos específicos, con movimientos y escenas acotados, creados para el uso académico como los datasets KTH y HMDB51
\parencite{schuldt2004recognizing, kuehne2011hmdb, whiten2013mofreak} y otros con clips pertenecientes a películas y deportes
\parencite{chen2011violence, nievas2011violence, deniz2014fast}. En ambos casos, los resultados fueron alentadores, mostrando el
potencial de esta manera de abordar la detección y clasificación.