\section{Fundamentos y estado del conocimiento sobre el tema} \label{fundyestarte}
\iffalse
Escriba una breve introducción general al tema y cite y comente las mayores
contribuciones en el tema específico, incluyendo bibliografıa actualizada.
\fi
La clasificación de imágenes o vídeos en categorías semánticas es un problema de interés tanto para la comunidad científica como
para la industria. La detección de diferentes tipos de escenas por lo general se basa en vectores de características que describen
el color y la textura de las imágenes entre otras propiedades visuales.

Hace ya más de una década, comenzó a verse una tendencia a usar \textit{keypoints} y puntos de interés local en la recuperación y
clasificación de la información contenida en las imágenes \parencite{csurka2004visual, lopes2010action}. Los \textit{keypoints} son zonas destacadas
de las imágenes, que contienen abundante información local acerca de la misma, los cuales pueden ser identificados usando diferentes detectores y representados
por diversos descriptores.

Una vez que los \textit{keypoints} son obtenidos, éstos son distribuidos en una gran cantidad de \textit{clusters}, asignando a un mismo
\textit{cluster} aquellos \textit{keypoints} de características similares. Cada \textit{cluster} es considerado una \textit{visual word}
que representa el patrón local específico compartido por todos los \textit{keypoints} de ese \textit{cluster}, lo que permite obtener un
vocabulario de \textit{visual words} que describe todos los patrones locales de las imágenes. A partir
de ésto, una imagen puede representarse como un \textit{Bag of Visual Words}, un vector que contiene
la cantidad de veces que cada \textit{visual word} aparece en la imagen, el cual es usado como vector
de características durante la clasificación \parencite{yang2007evaluating}.

Esta representación es análoga a la de Bag of Words utilizada en textos para describir tanto la forma
como la semántica. Ésto permite que muchas de las técnicas ya desarrolladas y usadas para el análisis de textos hayan podido ser aplicadas
al trabajo con imágenes.

Los \textit{keypoints} suelen encontrarse en los ángulos y bordes de los objetos presentes en las imágenes.
Uno de los métodos más utilizados para su detección es la Diferencia de Gaussianas (DoG) \parencite{lowe2004distinctive}, sin embargo existen otros como
Harris Corner Detector \parencite{harris1988combined}, Fast Hessian Detector \parencite{bay2006surf}, AGAST \parencite{mair2010adaptive} y
multi-scale AGAST \parencite{leutenegger2011brisk}. 

En cuanto a los descriptores, a lo largo de los años se han publicado numerosos desarrollos. Para trabajar con imágenes, algunos de los más utilizados
son SIFT \parencite{lowe2004distinctive}, SURF \parencite{bay2006surf}, BRISK \parencite{leutenegger2011brisk} y FREAK \parencite{alahi2012freak}.
Por lo general, cada descriptor se utiliza acompañado de un detector específico.
En lo referente a vídeos, suelen utilizarse descriptores que se basan en los de imágenes como MoSIFT \parencite{chen2009mosift} y
MoFREAK \parencite{whiten2013mofreak}. Éstos desglosan los vídeos en las imágenes que los componen (\textit{frames}), para luego analizar por separado
cada una de ellas. Con esta información ya disponible se incorpora el factor temporal, siendo éste el eje en la relación entre los distintos \textit{frames},
permitiendo, por ejemplo, analizar variaciones de un \textit{frame} a otro.


Una vez que se completa el proceso de obtención de información de los objetos a clasificar y se construye el \textit{Bag of Visual
Words}, se llega a la etapa de clasificación propiamente dicha. Para completar esta última etapa uno de los métodos más utilizado
actualmente es Support Vector Machine (SVM). La \textit{performance} del SVM puede variar en gran medida de acuerdo al \textit{kernel}
elegido, algunos de los más populares son: Histogram Intersection Kernel (HIK) \parencite{barla2003histogram}, Radial Basis Function (RBF) y Chi-Squared,
el cual es una variante de RBF \parencite{whiten2013mofreak, nievas2011violence}.

Se han hecho diversas pruebas en vídeos utilizando la idea de \textit{Bag of Visual Words}, algunas con
datasets de vídeos específicos, con movimientos y escenas acotados, creados para el uso académico como los datasets KTH \parencite{schuldt2004recognizing}
y HMDB51 \parencite{kuehne2011hmdb} \parencite{chen2009mosift, whiten2013mofreak} y otros con clips pertenecientes a películas \parencite{chen2011violence}, deportes
 \parencite{nievas2011violence, deniz2014fast} y cámaras de vigilancia \parencite{chen2009mosift}. En ambos casos, los resultados fueron alentadores, mostrando el
potencial de esta manera de abordar la detección y clasificación.