\section{Objetivos espec\'ificos}
%Enuncie de manera clara las metas concretas a alcanzar en el marco de la tesina.
Proponemos desarrollar una nueva herramienta que nos permita modelar
algoritmos y modelos paralelos, y observar en detalle la estructura
de una computaci\'on paralela. Para esto proponemos dise\~nar un EDSL
(lenguaje de dominio espec\'ifico embebido) en Haskell que capture las
computaciones paralelas y las simule. 
%
Debido a que el EDSL es embebido, no ser\'a necesario capturar las
computaciones secuenciales, las cuales ser\'an ejecutadas por Haskell
en forma usual.

% En contraposici\'on con ThreadScope,
% se buscar\'a desarrollar un DSL que nos permita modelar y observar el paralelismo
% que el desarrollador construy\'o, sin la necesidad de tener que
% ejecutar el programa. 

\iffalse
  \item[Desarrollo de sintaxis:] Se buscar\'a desarrollar una nueva sintaxis para
la modelaci\'on de computaciones paralelas, que permita al desarrollador
encapsular el paralelismo en forma que sea independiente a la computaci\'on.
  \item[Sem\'antica de dicha sintaxis:] Se establecer\'a la sem\'antica de dicha
sintaxis,
permitiendo al desarollador especificar computaciones paralelas
libremente
permitiendo que el desarrollador obtenga el control sobre todas las computaciones
paralelas, sin especializar el tipo de paralelismo.
  \item[Desarrollo de casos de estudio:] Se buscar\'an y desarrollar\'an casos
de estudio que permitan evaluar el rendimiento al utilizar la sintaxis desarrollada,
mostrando que la sintaxis permite describir paralelismo f\'acilmente.

\end{description}

\fi
