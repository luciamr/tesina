\section{Objetivos específicos}
%Enuncie de manera clara las metas concretas a alcanzar en el marco de la tesina.
Proponemos desarrollar una herramienta que nos permita identificar acciones específicas en videos correspondientes a partidos de Rugby.

Trabajaremos con un dataset compuesto por numerosos clips de corta duración, los cuales fueron obtenidos a partir de filmaciones de partidos
de Rugby. Los videos varían mucho entre sí, ya que se utilizaron tanto filmaciones amateur como transmiciones de televisión.

Éstos clips se clasifican en tres clases diferentes: \textit{line}, \textit{scrum} y \textit{juego}. Nuestro objetivo es poder automatizar su
reconocimiento y clasificación.

Para lograrlo nos basaremos en el modelo de \textit{Bag-of-Words}. Estudiaremos distintos detectores y descriptores con el fin de encontrar
los más adecuados para el dataset con el que estamos trabajando. También analisaremos y compararemos el costo computacional de ellos.

\iffalse
  \item[Desarrollo de sintaxis:] Se buscar\'a desarrollar una nueva sintaxis para
la modelaci\'on de computaciones paralelas, que permita al desarrollador
encapsular el paralelismo en forma que sea independiente a la computaci\'on.
  \item[Sem\'antica de dicha sintaxis:] Se establecer\'a la sem\'antica de dicha
sintaxis,
permitiendo al desarollador especificar computaciones paralelas
libremente
permitiendo que el desarrollador obtenga el control sobre todas las computaciones
paralelas, sin especializar el tipo de paralelismo.
  \item[Desarrollo de casos de estudio:] Se buscar\'an y desarrollar\'an casos
de estudio que permitan evaluar el rendimiento al utilizar la sintaxis desarrollada,
mostrando que la sintaxis permite describir paralelismo f\'acilmente.

\end{description}

\fi
