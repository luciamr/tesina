\section{Motivación}
\iffalse
Explique el problema o situación de referencia en el que se desarrolla la
propuesta o los interrogantes en el campo disciplinario a los que la propuesta se
dirige. Desarrolle la importancia e impacto de los objetivos y el conocimiento
que se generará. En esta sección no es necesario describir las tareas específicas
que se realizarán (para eso, ver Objetivos específicos).
\fi

%Análisis de videos
\iffalse
El desarrollo de nuevas tecnologías, acompañado de la reducción en el costo de las mismas, ha hecho que el acceso a medios para la 
producción de material audio-visual se haya masificado, cambiando rotundamente la forma en que se trabajaba con videos y generando nuevas
herramientas, así como nuevos desafíos y necesidades.

Todo esto ha permitido extender la utilización de cámaras de video a otros ámbitos más allá de la producción de contenidos de entretenimiento,
alcanzando distintos objetivos tales como el control de seguridad y el almacenamiento de recuerdos personales, entre otros.
Esto genera grandes volúmenes de datos conservados en formato de video. 
Si bien los mismos pueden contener muchísima información útil, su análisis y clasificación no es sencilla. En ámbitos profesionales,
ya sea para análisis de seguridad o contenidos de entretenimiento, la cantidad de material que se produce ha hecho que sea muy costoso y prácticamente
inviable analizarlos manualmente. Por esta razón, se busca automatizar el análisis de los mismos.

A lo largo de los últimos años, se han desarrollado distintas técnicas y enfoques para el análisis de videos.
Puesto que éstos no son otra cosa que secuencias de imágenes con sonido incorporado, uno de los enfoques para su análisis
se basa en la utilización de los métodos ya desarrollados para imágenes con la incorporación del factor temporal, el cual puede
aportar nueva información. Si bien el sonido presente también puede ayudar en el estudio, muchas veces es descartado durante el análisis,
basándose éste solamente en los aspectos visuales.
\fi

La producción de material audiovisual ha sufrido importantes cambios en las últimas décadas. La reducción de costos junto con la aparición de nuevas
tecnologías permitieron masificar el acceso a medios de grabación y almacenamiento, modificando rotundamente la forma en que se trabajaba con videos y
generando tanto nuevas herramientas como desafíos y necesidades.

Con todos estos cambios, las cámaras de video que originalmente habían sido concebidas para la producción de contenidos de entretenimientos pasaron a
formar parte de la vida cotidiana.

Ya sea a través de videograbadoras o usando cámaras integradas en otros dispositivos como celulares, computadoras o cámaras de fotos, los videos
comenzaron a utilizarse con nuevos fines, incluídos monitoreo, control de seguridad, comunicación, medicina, almacenamiento de recuerdos personales,
y educación entre otros.

Ésto trajo aparejado un incremento exponencial en la cantidad de información generada día a día, haciendo inviable el reconocimiento manual de las
imágenes obtenidas. Como consecuencia de ésto, los esfuerzos por automatizar la detección de eventos en videos crecieron, ya que abrían un abanico
de nuevas posibles aplicaciones, tanto para videos obtenidos previamente como para el análisis en tiempo real.

En este contexto, el estudio del movimiento humano tuvo un lugar muy importante. Sin embargo, esta tarea no es simple ya que hay importantes disparidades
en los movimientos, variaciones entre los distintos escenarios y grandes diferencias entre individuos. Para superar estas dificultades, se ha trabajado
desde distintos enfoques y con distintos algoritmos, siempre buscando el balance entre precisión y costo computacional.

Los primeros trabajos fueron realizados con videos especialmente creados para este fin, donde había un único individuo realizando movimientos repetitivos
frente a la cámara y el fondo era prácticamente liso para que no interfiriera en el análisis, como KTH\parencite{schuldt2004recognizing}. Una vez que hubo
más avances, se empezaron a estudiar conjuntos de videos más complejos, donde entraban en juego nuevos desafíos, como cambios de cámara, escenarios
dinámicos, multiples objetos por cuadro, diferencias de luz y escala, entre otros, como los trabajos realizados en la Universidad de Castilla-La
Mancha\parencite{nievas2011violence, deniz2014fast}.

Los resultados obtenidos en estos trabajos confirman los avances que hubo en las áreas de reconocimiento y clasificación en las últimas décadas.




%EL reconocimiento de acciones humanas basado en visión consiste en el
%proceso de etiquetado de secuencias con la acción correspondiente. *** En 

\iffalse
%Utilización de descriptores
Un método común y que ha resultado ser eficiente para obtener información a partir de imágenes es la utilización de descriptores.
Estos descriptores informan acerca de las características visuales de la imagen, describiendo características elementales como la forma, el color,
la textura y la ubicación de elementos dentro de la misma.
Para extender esta idea al plano de los videos, lo que se hace es desglosar el video en las imágenes que lo componen (frames), para luego analizar por separado
cada una de ellas. Con esta información ya disponible se incorpora el factor temporal, siendo éste el eje en la relación entre los distintos frames, permitiendo,
por ejemplo, analizar variaciones de un frame a otro.
\fi
