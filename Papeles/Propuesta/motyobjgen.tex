\section{Motivación y Objetivo General}
\iffalse
Explique el problema o situación de referencia en el que se desarrolla la
propuesta o los interrogantes en el campo disciplinario a los que la propuesta se
dirige. Desarrolle la importancia e impacto de los objetivos y el conocimiento
que se generará. En esta sección no es necesario describir las tareas específicas
que se realizarán (para eso, ver Objetivos específicos).
\fi

%Análisis de videos
Como consecuencia del desarrollo de nuevas tecnologías y la reducción de costos de la que estuvo acompañada, el acceso a medios para la 
producción de material audio-visual se ha masificado, cambiando rotundamente la forma en que se trabajaba con videos y generando nuevas
herramientas, así como nuevos desafíos y necesidades.

Todo esto a permitido extender la utilización de cámaras de video a otros ámbitos más allá de la producción de contenidos de entretenimiento,
alcanzando distintos objetivos como control de seguridad y recuerdos personales entre otros. Esto genera grandes volúmenes de datos.
Si bien estos videos pueden contener muchísima información útil, su análisis y clasificación no es sencilla. En ámbitos profesionales,
ya sea para análisis de seguridad o contenidos de entretenimiento, la cantidad de material que se produce ha hecho que sea muy costoso y prácticamente
inviable analizarlos manualmente. Por está razón, se busca automatizar el análisis de los mismos.

A lo largo de los últimos años, se han desarrolado distintas técnicas y enfoques para el análisis de videos.
Puesto que éstos no son otra cosa que secuencias de imágenes con sonido incorporado, uno de los enfoques para su análisis
se basa en la utilización de los métodos ya desarrollados para imágenes con la incorporación del factor temporal, el cual puede
aportar nueva información. Si bien el sonido presente también puede ayudar en el estudio, muchas veces es descartado durante el análisis,
basándose éste solamente en los aspectos visuales.

%Utilización de descriptores
Un método común, y que ha resultado ser eficiente, para obtener información a partir de imágenes es la utilización de descriptores.
Estos descriptores informan acerca de las características visuales de la imagen, describiendo características elementales como la forma, el color,
la textura y la ubicación de elementos dentro de la imagen.
Para extender esta idea al plano de los videos, lo que se hace es desglosar el video en las imágenes que lo componen (frames), para luego analizar por separado
cada una de ellas. Con esta información ya disponible se incorpora el factor temporal, siendo el eje en la relación entre los distintos frames, permitiendo,
por ejemplo, analizar variaciones de un frame a otro.
