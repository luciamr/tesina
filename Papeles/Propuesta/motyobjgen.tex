\section{Motivación y Objetivo General}
\iffalse
Explique el problema o situación de referencia en el que se desarrolla la
propuesta o los interrogantes en el campo disciplinario a los que la propuesta se
dirige. Desarrolle la importancia e impacto de los objetivos y el conocimiento
que se generará. En esta sección no es necesario describir las tareas específicas
que se realizarán (para eso, ver Objetivos específicos).
\fi

%Análisis de videos
A partir de la reducción de costos en la blabla la creación de videos se hizo masiva. Esto genera grandes volumenes de datos.
Imposible de analizar manualmente. Se busca automatizar el análisis.

Puesto que los videos no son otra cosa que secuencias de imágenes con sonido incorporado, uno de los enfoques para su análisis
se basa en la utilización de los métodos ya desarrollados para imágenes y la incoorporación del factor temporal. 



As a result of the new communication technologies and the massive use of Internet in our society, the amount of audio-visual information available in digital format is increasing considerably. Therefore, it has been necessary to design some systems that allow us to describe the content of several types of multimedia information in order to search and classify them.
The audio-visual descriptors are in charge of the contents description. These descriptors have a good knowledge of the objects and events found in a video, image or audio and they allow the quick and efficient searches of the audio-visual content.
This system can be compared to the search engines for textual contents. Although it is certain, that it is relatively easy to find text with a computer, is much more difficult to find concrete audio and video parts. For instance, imagine somebody searching a scene of a happy person. The happiness is a feeling and it is not evident its shape, color and texture description in images.
The description of the audio-visual content is not a superficial task and it is essential for the effective use of this type of archives. The standardization system that deals with audio-visual descriptors is the MPEG-7 (Motion Picture Expert Group - 7).
Types of visual descriptors[edit]
Descriptors are the first step to find out the connection between pixels contained in a digital image and what humans recall after having observed an image or a group of images after some minutes.
Visual descriptors are divided in two main groups:
General information descriptors: they contain low level descriptors which give a description about color, shape, regions, textures and motion.
Specific domain information descriptors: they give information about objects and events in the scene. A concrete example would be face recognition.
General information descriptors[edit]
General information descriptors consist of a set of descriptors that covers different basic and elementary features like: color, texture, shape, motion, location and others. This description is automatically generated by means of signal processing.

%Utilización de descriptores
Un método común para obtener información a partir de imágenes es la utilización de descriptores. Estos descriptores informan acerca de las características
visuales de la imagen, describiendo características elementales como la forma, el color o la textura.
Para extender esta idea al plano de los videos, lo que se hace es desglosar el video en las imágenes que lo componen (frames), para luego analizar por separado
cada una de estas imagenes. Con esta información ya disponible se incorpora el factor temporal, siendo el eje en la relación entre los distintos frames, permitiendo,
por ejemplo, analizar cambios de un frame a otro.

% Problema con Multicores
Si bien aumentando los n\'ucleos dentro de los procesadores nos
permite en \textit{teor\'ia} poder ejecutar instrucciones en paralelo
multiplicando el poder de c\'omputo, en la pr\'actica no es tan
f\'acil obtener un mayor rendimiento.  Cuando se aumentaba la
frecuencia de los microprocesadores se ejecutaban \textbf{exactamente
  las mismas} instrucciones. Es decir, se ejecutaban m\'as r\'apido
\textbf{los mismos algoritmos}. En cambio, en un procesador
multin\'ucleo, los algoritmos deber\'an tener en cuenta c\'omo se van
a distribuir las tareas a los distintos n\'ucleos y c\'uales tareas se
podr\'an realizar en paralelo. Es decir, se deber\'an adaptar
los algoritmos para decidir qu\'e tareas se podr\'an desarrollar en
paralelo, e incluso se deber\'an adaptar los lenguajes de
programaci\'on para dotar a los programadores con primitivas lo
suficientemente expresivas para que puedan explotar todo el hardware
subyacente. Por su parte, los compiladores se deber\'an modificar para
que puedan generar ejecutables que manejen la comunicaci\'on entre
n\'ucleos, manejo de memoria, evitar o detectar \textit{deadlocks},
etc.

Para incorporar paralelismo en los programas se han extendido los
lenguajes de programaci\'on, o bien se han desarrollado bibliotecas,
con diversos grados de \'exito.  El estudio de formas de incorporar
en forma simple y eficiente la programaci\'on paralela al desarrollo
de software sigue siendo objeto de investigaci\'on.  En particular, en
el lenguaje de programaci\'on Haskell se han desarrollado una variedad
de extensiones del lenguaje, bibliotecas, y abstracciones, que permiten
encarar el problema de la programaci\'on paralela desde diversos
\'angulos.

Para dar soporte a la experimentaci\'on con formas nuevas de
incorporar paralelismo al desarrollo de software es necesario el
desarrollo de herramientas adecuadas que permitan el an\'alisis de las
ejecuciones obtenidas.
%Objetivo general?
El objetivo general de esta tesina ser\'a el desarrollo de un lenguaje
de dominio espec\'ifico embebido en Haskell que permita la
simulaci\'on simb\'olica de programas paralelos escritos para varias
de las abstracciones existentes.
