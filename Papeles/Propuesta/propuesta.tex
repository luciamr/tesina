\documentclass[11pt]{article}
\usepackage{a4wide}
%\usepackage[cm]{fullpage} si lo queremos mas ancho

%biblatex
%\usepackage[url=false, doi=false, style=authoryear, uniquename=false]{biblatex}
%\usepackage[style=authoryear, uniquename=false]{biblatex}
\usepackage{natbib}
%\bibliography{cites}   
%\renewcommand*{\bibfont}{\footnotesize}
%\renewcommand*{\bibfont}{\small}

\usepackage[spanish,activeacute]{babel}
\usepackage[utf8]{inputenc}
\usepackage{enumerate}
\usepackage{todonotes}
\usepackage{listings}
\usepackage{stmaryrd}

\begin{document}
\date{9 de diciembre de 2015}
\title{Propuesta de Tesina para la obtención del grado \\ Licenciado en Ciencias de la Computación}
\maketitle

\begin{description}
  \item[Postulante:] Martín Reixach, Lucía Norma
  \item[Director:] Granitto, Pablo
  \item[Codirector:] Larese, Mónica
\end{description}

\section{Situación del postulante}
Resta aprobar las materias Sistemas Operativos, Inteligencia Artificial y Compiladores,
planeo completarlas durante el primer cuatrimestre del próximo año.
Actualmente estoy realizando actividad docente en las materias Programación I y Programación II.
Se espera dedicar al menos 25 hs. semanales a la realización de la Tesina.

\section{Título}
Detección de eventos en videos: análisis de partidos de Rugby.
%\todo{Poner bien el titulo.}

\section{Motivación y Objetivo General}
\iffalse
Explique el problema o situación de referencia en el que se desarrolla la
propuesta o los interrogantes en el campo disciplinario a los que la propuesta se
dirige. Desarrolle la importancia e impacto de los objetivos y el conocimiento
que se generará. En esta sección no es necesario describir las tareas específicas
que se realizarán (para eso, ver Objetivos específicos).
\fi

%Procesamiento de videos

%Historia de los micros

Desde su invenci\'on en 1971, los procesadores fueron evolucionando
para aumentar su poder de c\'omputo.
%Freq Scaling & Multicores
Para poder lograrlo se utiliz\'o una t\'ecnica denominada
\textbf{transistor-speed scaling}, que permit\'ia que en cada
generaci\'on de procesadores la velocidad de c\'omputo se duplicara,
disminuyendo el tama\~no de los transistores. Sin embargo, al llegar a
tama\~nos muy peque\~nos, los transistores pierden energ\'ia y como
consecuencia se aumenta tanto el consumo que la t\'ecnica ya no es
viable~\parencite{Borkar:2011:FM:1941487.1941507}. Como forma
alternativa de aumentar la capacidad de procesamiento, se opt\'o por
retomar la idea de aumentar la cantidad de unidades de procesamiento
dentro de un mismo
procesador~\parencite{Noguchi:1975:DCH:1499949.1500062}.

% Problema con Multicores
Si bien aumentando los n\'ucleos dentro de los procesadores nos
permite en \textit{teor\'ia} poder ejecutar instrucciones en paralelo
multiplicando el poder de c\'omputo, en la pr\'actica no es tan
f\'acil obtener un mayor rendimiento.  Cuando se aumentaba la
frecuencia de los microprocesadores se ejecutaban \textbf{exactamente
  las mismas} instrucciones. Es decir, se ejecutaban m\'as r\'apido
\textbf{los mismos algoritmos}. En cambio, en un procesador
multin\'ucleo, los algoritmos deber\'an tener en cuenta c\'omo se van
a distribuir las tareas a los distintos n\'ucleos y c\'uales tareas se
podr\'an realizar en paralelo. Es decir, se deber\'an adaptar
los algoritmos para decidir qu\'e tareas se podr\'an desarrollar en
paralelo, e incluso se deber\'an adaptar los lenguajes de
programaci\'on para dotar a los programadores con primitivas lo
suficientemente expresivas para que puedan explotar todo el hardware
subyacente. Por su parte, los compiladores se deber\'an modificar para
que puedan generar ejecutables que manejen la comunicaci\'on entre
n\'ucleos, manejo de memoria, evitar o detectar \textit{deadlocks},
etc.

Para incorporar paralelismo en los programas se han extendido los
lenguajes de programaci\'on, o bien se han desarrollado bibliotecas,
con diversos grados de \'exito.  El estudio de formas de incorporar
en forma simple y eficiente la programaci\'on paralela al desarrollo
de software sigue siendo objeto de investigaci\'on.  En particular, en
el lenguaje de programaci\'on Haskell se han desarrollado una variedad
de extensiones del lenguaje, bibliotecas, y abstracciones, que permiten
encarar el problema de la programaci\'on paralela desde diversos
\'angulos.

Para dar soporte a la experimentaci\'on con formas nuevas de
incorporar paralelismo al desarrollo de software es necesario el
desarrollo de herramientas adecuadas que permitan el an\'alisis de las
ejecuciones obtenidas.
%Objetivo general?
El objetivo general de esta tesina ser\'a el desarrollo de un lenguaje
de dominio espec\'ifico embebido en Haskell que permita la
simulaci\'on simb\'olica de programas paralelos escritos para varias
de las abstracciones existentes.

\section{Fundamentos y estado del conocimiento sobre el tema}
\iffalse
Escriba una breve introducción general al tema y cite y comente las mayores
contribuciones en el tema específico, incluyendo bibliografıa actualizada.
\fi
La clasificación de imágenes o videos en categorías semánticas en un problema de interés tanto para la actividad científica como
para la industria. La detección de diferentes tipos de escenas por lo general se basa en vectores de características que describen
el color y la textura entre otras propiedades visuales de las imágenes.

Hace ya más de una década, comenzó a verse una tendencia a usar \textit{keypoints} y puntos de interés local en la recuperación y
clasificación de la información contendida en la imágenes. ***** Los \textit{keypoints} son zonas destacadas de las imágenes, que contienen
abundante información local acerca de la misma, los cuales pueden ser identificados usando diferentes detectores *** y respresentados
por diversos descriptores.***

Una vez que los \textit{keypoints} son obtenidos, éstos son distribuidos en una gran cantidad de \textit{clusters}, asignando a un mismo
\textit{cluster} aquellos de características similares. Cada \textit{cluster} es considerado una \textit{\textquotedblleft visual word\textquotedblright}
que representa el patrón local específico compartido por todos los \textit{keypoints} de ese \textit{cluster}, ésto permite obtener un
vocabulario de \textit{\textquotedblleft visual words\textquotedblright} que describe todos los patrones locales de las imágenes. A partir
de ésto, una imagen puede respresentarse como una \textit{\textquotedblleft bag of visual words\textquotedblright}, un vector que contiene
la cantidad de veces que cada \textit{\textquotedblleft visual word\textquotedblright} aparece en la imagen, el cual es usado como vector
de características durante la clasificación.

Ésta representación es análoga a la de \textquotedblleft bag of words\textquotedblright, utilizada en textos para describir tanto la forma
como la semántica. Ésto permite que muchas de las técnicas ya desarrolladas y usadas para el análisis de textos hayan podido ser aplicadas
al trabajo con imágenes.

Uno de los métodos más utilizados para la detección de \textit{keypoints} es la Diferencia Gaussiana (DoG) .... Los \textit{keypoints} suelen
encontrarse en los ángulos y bordes de los objetos presentes en las imágenes. BRISK ver

En cuanto a los descriptores, a lo largo de los años se han publicado numerosos desarrollos. Para trabajar con imágenes, alguno de los más utilizados
son SIFT y FREAK. En lo referente a videos, suelen utilizarse descriptores que se basan en los de imágenes e incorporan el factor temporal,
como MoSIFT y MoFREAK.

Una vez que se completa el proceso de obtención de información de los objetos a clasificar y construimos nuestro \textit{\textquotedblleft bag of visual
words\textquotedblright}, se llega a la etapa de clasificación propiamente dicha. Para completar esta última etapa uno de los métodos más utilizado
actualmente es \textit{Support Vector Machine} (SVM).

Se han hecho diversas pruebas en videos utilizando la idea de \textit{\textquotedblleft bag of visual words\textquotedblright}, algunas con
datasets de videos específicos, con movimientos y escenas acotados, creados para el uso académico *** y otros con clips pertenecientes a películas
y deportes.*** En ambos casos, los resultados fueron alentadores, mostrando el potencial de esta manera de abordar la detección y clasificación.
\section{Objetivos específicos}
%Enuncie de manera clara las metas concretas a alcanzar en el marco de la tesina.
Proponemos desarrollar una herramienta que nos permita identificar \textit{lines}, \textit{scrums} y
\textit{momentos de juego} en videos correspondientes a partidos de Rugby.

Para lograrlo nos basaremos en el método de Bag-of-Words junto a descriptores de imágenes y videos
(en este trabajo el audio del video no será tenido en cuenta).

También se analisará la \textit{performance} de los métodos de detección usados y se compararán los
resultados obtenidos con los alcanzados utilizando otros métodos.

\iffalse
  \item[Desarrollo de sintaxis:] Se buscar\'a desarrollar una nueva sintaxis para
la modelaci\'on de computaciones paralelas, que permita al desarrollador
encapsular el paralelismo en forma que sea independiente a la computaci\'on.
  \item[Sem\'antica de dicha sintaxis:] Se establecer\'a la sem\'antica de dicha
sintaxis,
permitiendo al desarollador especificar computaciones paralelas
libremente
permitiendo que el desarrollador obtenga el control sobre todas las computaciones
paralelas, sin especializar el tipo de paralelismo.
  \item[Desarrollo de casos de estudio:] Se buscar\'an y desarrollar\'an casos
de estudio que permitan evaluar el rendimiento al utilizar la sintaxis desarrollada,
mostrando que la sintaxis permite describir paralelismo f\'acilmente.

\end{description}

\fi

\section{Metodología y Plan de Trabajo}
\iffalse
Se recomienda estructurar esta sección en función de los objetivos específicos.
* Planteo de la hipotesis a analizar en cada objetivo o seccion del proyecto.
* Actividades propuestas y metodologıa a usar en cada una de ellas.
* Resultados que se esperan obtener o metas a cumplir y como se evaluaran
los resultados.
Trate de evaluar los potenciales problemas y limitaciones de la metodolog ́ıa
y t ́ecnicas propuestas y en lo posible proponer alternativas.
\fi

Para alcanzar los objetivos propuestos se proponen las siguientes tareas

Programa tentativo de trabajo:
\begin{itemize}
  \item Estudio de las abstracciones para programaci\'on paralela en Haskell: 4 semanas.
  \item Dise\~no del DSL: 8 semanas. El dise\~no se har\'a en forma
    progresiva, empezando por un DSL simple que capture las
    operaciones b\'asicas, para luego ir
    agreg\'andole caracter\'isticas m\'as avanzadas (extensi\'on para
    m\'onada Eval, distinci\'on entre computaciones y valores,
    representaci\'on gr\'afica).
  \item Desarrollo de casos de prueba y evaluaci\'on de los resultados: 4 semanas.
  \item S\'intesis de los resultados obtenidos y escritura de la tesina: 6 semanas.
\end{itemize}
El trabajo se realizará durante aproximadamente 4 meses con una
dedicación media de trabajo.



%\printbibliography
\bibliographystyle{abbrv} %siam
\nocite{*}
\bibliography{cites}

\end{document}
