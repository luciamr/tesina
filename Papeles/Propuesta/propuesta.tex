\documentclass[11pt]{article}
\usepackage{a4wide}
%\usepackage[cm]{fullpage} si lo queremos mas ancho

%biblatex
%\usepackage[url=false, doi=false, style=authoryear, uniquename=false]{biblatex}
%\usepackage[style=authoryear, uniquename=false]{biblatex}
\usepackage{natbib}
%\bibliography{cites}   
%\renewcommand*{\bibfont}{\footnotesize}
%\renewcommand*{\bibfont}{\small}

\usepackage[spanish,activeacute]{babel}
\usepackage[utf8]{inputenc}
\usepackage{enumerate}
\usepackage{todonotes}
\usepackage{listings}
\usepackage{stmaryrd}

\begin{document}
\date{9 de diciembre de 2015}
\title{Propuesta de Tesina para la obtención del grado \\ Licenciado en Ciencias de la Computación}
\maketitle

\begin{description}
  \item[Postulante:] Martín Reixach, Lucía Norma
  \item[Director:] Granitto, Pablo
  \item[Codirector:] Larese, Mónica
\end{description}

\section{Situación del postulante}
Resta aprobar las materias Sistemas Operativos, Inteligencia Artificial y Compiladores,
planeo completarlas durante el primer cuatrimestre del próximo año.
Actualmente estoy realizando actividad docente en las materias Programación I y Programación II.
Se espera dedicar al menos 25 hs. semanales a la realización de la Tesina.

\section{Título}
Detección de eventos en videos: análisis de partidos de Rugby.
%\todo{Poner bien el titulo.}

\section{Motivación y Objetivo General}
\iffalse
Explique el problema o situación de referencia en el que se desarrolla la
propuesta o los interrogantes en el campo disciplinario a los que la propuesta se
dirige. Desarrolle la importancia e impacto de los objetivos y el conocimiento
que se generará. En esta sección no es necesario describir las tareas específicas
que se realizarán (para eso, ver Objetivos específicos).
\fi

%Análisis de videos
El desarrollo de nuevas tecnologías, acompañado de la reducción de costo de las mismas, ha hecho que el acceso a medios para la 
producción de material audio-visual se haya masificado, cambiando rotundamente la forma en que se trabajaba con videos y generando nuevas
herramientas, así como nuevos desafíos y necesidades.

Todo esto ha permitido extender la utilización de cámaras de video a otros ámbitos más allá de la producción de contenidos de entretenimiento,
alcanzando distintos objetivos tales como el control de seguridad y el almacenamiento de recuerdos personales, entre otros.
Esto genera grandes volúmenes de datos conservados en formato de video. 
Si bien los mismos pueden contener muchísima información útil, su análisis y clasificación no es sencilla. En ámbitos profesionales,
ya sea para análisis de seguridad o contenidos de entretenimiento, la cantidad de material que se produce ha hecho que sea muy costoso y prácticamente
inviable analizarlos manualmente. Por esta razón, se busca automatizar el análisis de los mismos.

A lo largo de los últimos años, se han desarrollado distintas técnicas y enfoques para el análisis de videos.
Puesto que éstos no son otra cosa que secuencias de imágenes con sonido incorporado, uno de los enfoques para su análisis
se basa en la utilización de los métodos ya desarrollados para imágenes con la incorporación del factor temporal, el cual puede
aportar nueva información. Si bien el sonido presente también puede ayudar en el estudio, muchas veces es descartado durante el análisis,
basándose éste solamente en los aspectos visuales.

La producción de material audiovisual ha sufrido importantes cambios en las últimas décadas. La reducción de cosotos junto con la aparición de nuevas
tecnologías permitieron masificar el acceso a medios de grabación y almacenamiento, modificando rotundamente la forma en que se trabajaba con videos y
generando tanto nuevas herramientas como desafíos y necesidades.

Con todos estos cambios las cámaras de video que originalmente habían sido concebidas para la producción de contenidos de entretenimientos pasaron a
formar parte de la vida cotidiana.

Ya sea a través de videograbadoras o usando cámaras integradas en otros dispositivos como celulares, computadoras o cámaras de fotos, los videos
comenzaron a utilizarse con nuevos fines, incluídos monitoreo, control de seguridad, comunicación, medicina, almacenamiento de recuerdos personales,
y educación entre otros.

Esto trajo aparejado un incremento exponencial en la cantidad de información generada día a día, haciendo cada vez más inviable el análisis manual de
las imágenes obtenidas. Como consecuencia de ésto, los esfuerzos por automatizar la detección de eventos en videos crecieron, ya que abrían un abanico
de nuevas posibles aplicaciones, tanto diferidas como en tiempo real.

En este contexto, el estudio del movimiento humano tuvo un lugar muy importante. EL reconocimiento de acciones humanas basado en visión consiste en el
proceso de etiquetado de secuencias con la acción correspondiente. *** Esta tarea no es simple ya que hay importantes variaciones en los movimientos,
modificaciones entre los distintos escenarios y grandes diferencias entre individuos. Para superar estas dificultades, se ha trabajado desde distintos
enfoques y con distintos algoritmos, buscando la mayor precisión posible.

%Utilización de descriptores
Un método común y que ha resultado ser eficiente para obtener información a partir de imágenes es la utilización de descriptores.
Estos descriptores informan acerca de las características visuales de la imagen, describiendo características elementales como la forma, el color,
la textura y la ubicación de elementos dentro de la misma.
Para extender esta idea al plano de los videos, lo que se hace es desglosar el video en las imágenes que lo componen (frames), para luego analizar por separado
cada una de ellas. Con esta información ya disponible se incorpora el factor temporal, siendo éste el eje en la relación entre los distintos frames, permitiendo,
por ejemplo, analizar variaciones de un frame a otro.

\section{Fundamentos y estado del conocimiento sobre el tema}
\iffalse
Escriba una breve introducción general al tema y cite y comente las mayores
contribuciones en el tema específico, incluyendo bibliografıa actualizada.
\fi
La clasificación de imágenes o videos en categorías semánticas es un problema de interés tanto para la comunidad científica como
para la industria. La detección de diferentes tipos de escenas por lo general se basa en vectores de características que describen
el color y la textura de las imágenes entre otras propiedades visuales.

Hace ya más de una década, comenzó a verse una tendencia a usar \textit{keypoints} y puntos de interés local en la recuperación y
clasificación de la información contendida en la imágenes. ***** Los \textit{keypoints} son zonas destacadas de las imágenes, que contienen
abundante información local acerca de la misma, los cuales pueden ser identificados usando diferentes detectores *** y respresentados
por diversos descriptores.***

Una vez que los \textit{keypoints} son obtenidos, éstos son distribuidos en una gran cantidad de \textit{clusters}, asignando a un mismo
\textit{cluster} aquellos de características similares. Cada \textit{cluster} es considerado una \textit{visual word}
que representa el patrón local específico compartido por todos los \textit{keypoints} de ese \textit{cluster}, ésto permite obtener un
vocabulario de \textit{visual words} que describe todos los patrones locales de las imágenes. A partir
de ésto, una imagen puede respresentarse como una \textit{Bag of Visual Words}, un vector que contiene
la cantidad de veces que cada \textit{visual word} aparece en la imagen, el cual es usado como vector
de características durante la clasificación.

Ésta representación es análoga a la de Bag of Words utilizada en textos para describir tanto la forma
como la semántica. Ésto permite que muchas de las técnicas ya desarrolladas y usadas para el análisis de textos hayan podido ser aplicadas
al trabajo con imágenes.

Los \textit{keypoints} suelen encontrarse en los ángulos y bordes de los objetos presentes en las imágenes.
Uno de los métodos más utilizados para la detección de los \textit{keypoints} es la Diferencia Gaussiana (DoG), sin embargo existen otros como
Harris corner detector, Fast Hessian detector, AGAST y multi-scale AGAST. 

En cuanto a los descriptores, a lo largo de los años se han publicado numerosos desarrollos. Para trabajar con imágenes, alguno de los más utilizados
son SIFT\parencite{lowe2004distinctive} y FREAK\parencite{alahi2012freak}. Por lo general, cada descriptor se utiliza en conjunto a un detector
específico. En lo referente a videos, suelen utilizarse descriptores que se basan en los de imágenes e incorporan el factor temporal,
como MoSIFT\parencite{chen2009mosift} y MoFREAK\parencite{whiten2013mofreak}.

Una vez que se completa el proceso de obtención de información de los objetos a clasificar y construimos nuestro \textit{Bag of Visual
Words}, se llega a la etapa de clasificación propiamente dicha. Para completar esta última etapa uno de los métodos más utilizado
actualmente es \textit{Support Vector Machine} (SVM).

Se han hecho diversas pruebas en videos utilizando la idea de \textit{Bag of Visual Words}, algunas con
datasets de videos específicos, con movimientos y escenas acotados, creados para el uso académico como los datasets KTH y HMDB51
\parencite{schuldt2004recognizing, kuehne2011hmdb, whiten2013mofreak} y otros con clips pertenecientes a películas y deportes
\parencite{chen2011violence, nievas2011violence, deniz2014fast}. En ambos casos, los resultados fueron alentadores, mostrando el
potencial de esta manera de abordar la detección y clasificación.
\section{Objetivos específicos}
%Enuncie de manera clara las metas concretas a alcanzar en el marco de la tesina.
Proponemos desarrollar una herramienta que nos permita identificar acciones específicas en videos correspondientes a partidos de Rugby.

Trabajaremos con un dataset compuesto por numerosos clips de corta duración, los cuales fueron obtenidos a partir de filmaciones de partidos
de Rugby. Los videos varían mucho entre sí, ya que se utilizaron tanto filmaciones amateur como transmiciones de televisión. Todos los videos
se encuentran en formato .mp4 y tienen entre 24 y 30 FPS. Debido a la variedad en su origen, el tamaño y el aspecto varía entre ellos.

Éstos clips se clasifican en tres clases diferentes: \textit{line}, \textit{scrum} y \textit{juego}. Nuestro objetivo es poder automatizar su
reconocimiento y clasificación.

Para lograrlo nos basaremos en el modelo de \textit{Bag-of-Words}. Estudiaremos distintos detectores y descriptores con el fin de encontrar
los más adecuados para el dataset con el que estamos trabajando. También analisaremos y compararemos el costo computacional de ellos.

\iffalse
  \item[Desarrollo de sintaxis:] Se buscar\'a desarrollar una nueva sintaxis para
la modelaci\'on de computaciones paralelas, que permita al desarrollador
encapsular el paralelismo en forma que sea independiente a la computaci\'on.
  \item[Sem\'antica de dicha sintaxis:] Se establecer\'a la sem\'antica de dicha
sintaxis,
permitiendo al desarollador especificar computaciones paralelas
libremente
permitiendo que el desarrollador obtenga el control sobre todas las computaciones
paralelas, sin especializar el tipo de paralelismo.
  \item[Desarrollo de casos de estudio:] Se buscar\'an y desarrollar\'an casos
de estudio que permitan evaluar el rendimiento al utilizar la sintaxis desarrollada,
mostrando que la sintaxis permite describir paralelismo f\'acilmente.

\end{description}

\fi

\section{Plan de Trabajo}
\iffalse
Se recomienda estructurar esta sección en función de los objetivos específicos.
* Planteo de la hipotesis a analizar en cada objetivo o seccion del proyecto.
* Actividades propuestas y metodologıa a usar en cada una de ellas.
* Resultados que se esperan obtener o metas a cumplir y como se evaluaran
los resultados.
Trate de evaluar los potenciales problemas y limitaciones de la metodolog ́ıa
y t ́ecnicas propuestas y en lo posible proponer alternativas.
\fi

Para alcanzar los objetivos planteados proponemos las siguientes tareas (programa tentativo de trabajo):
\begin{itemize}
  \item Estudio de los métodos más usados para análisis de vídeos e imágenes. \textit{4 semanas}
  \item Implementación de los algoritmos necesarios. \textit{3 semanas}
    \\ La implementación se hará de manera progresiva, incorporando las herramientas que
    sean necesarias para obtener información más clara.
  \item Evaluación del funcionamiento de los métodos sobre los vídeos de nuestra base de datos y
  ajuste de los parámetros según sea necesario. \textit{3 semanas}
  \item Síntesis de los resultados obtenidos y escritura de la tesina. \textit{4 semanas}
\end{itemize}
El trabajo se realizará durante aproximadamente 3 meses con una dedicación de 30 hs. semanales.



%\printbibliography
\bibliographystyle{abbrv} %siam
\nocite{*}
\bibliography{cites}

\end{document}
