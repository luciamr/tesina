\section{Metodología y Plan de Trabajo}
\iffalse
Se recomienda estructurar esta sección en función de los objetivos específicos.
* Planteo de la hipotesis a analizar en cada objetivo o seccion del proyecto.
* Actividades propuestas y metodologıa a usar en cada una de ellas.
* Resultados que se esperan obtener o metas a cumplir y como se evaluaran
los resultados.
Trate de evaluar los potenciales problemas y limitaciones de la metodolog ́ıa
y t ́ecnicas propuestas y en lo posible proponer alternativas.
\fi

Para alcanzar los objetivos propuestos se proponen las siguientes tareas

Programa tentativo de trabajo:
\begin{itemize}
  \item Estudio de las abstracciones para programaci\'on paralela en Haskell: 4 semanas.
  \item Dise\~no del DSL: 8 semanas. El dise\~no se har\'a en forma
    progresiva, empezando por un DSL simple que capture las
    operaciones b\'asicas, para luego ir
    agreg\'andole caracter\'isticas m\'as avanzadas (extensi\'on para
    m\'onada Eval, distinci\'on entre computaciones y valores,
    representaci\'on gr\'afica).
  \item Desarrollo de casos de prueba y evaluaci\'on de los resultados: 4 semanas.
  \item S\'intesis de los resultados obtenidos y escritura de la tesina: 6 semanas.
\end{itemize}
El trabajo se realizará durante aproximadamente 4 meses con una
dedicación media de trabajo.

