\section{Plan de Trabajo}
\iffalse
Se recomienda estructurar esta sección en función de los objetivos específicos.
* Planteo de la hipotesis a analizar en cada objetivo o seccion del proyecto.
* Actividades propuestas y metodologıa a usar en cada una de ellas.
* Resultados que se esperan obtener o metas a cumplir y como se evaluaran
los resultados.
Trate de evaluar los potenciales problemas y limitaciones de la metodolog ́ıa
y t ́ecnicas propuestas y en lo posible proponer alternativas.
\fi

Para alcanzar los objetivos propuestos proponemos las siguientes tareas (programa tentativo de trabajo):
\begin{itemize}
  \item Estudio de los métodos más usados para análisis de videos e imágenes. \textit{4 semanas}
  \item Implementación de los algoritmos necesarios. \textit{3 semanas}
    \\ La implementación se hará de manera progresiva, incorporando las herramientas que
    sean necesarias para obtener información más clara.
  \item Evaluación del funcionamiento de los métodos sobre los videos de nuestra base de datos y
  ajuste de los parámetros según sea necesario. \textit{3 semanas}
  \item Síntesis de los resultados obtenidos y escritura de la tesina. \textit{4 semanas}
\end{itemize}
El trabajo se realizará durante aproximadamente 3 meses con una dedicación de 30 hs. semanales.

