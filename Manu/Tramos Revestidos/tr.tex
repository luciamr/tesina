\documentclass[10.5pt]{article}
\usepackage{a4wide}
%\usepackage[cm]{fullpage} si lo queremos mas ancho

\usepackage[backend=biber, style=numeric, uniquename=false]{biblatex}
%style=authoryear

\usepackage[spanish,activeacute]{babel}
\usepackage[utf8]{inputenc}
\usepackage{enumerate}
\usepackage{todonotes}
\usepackage{listings}
\usepackage{verbatim}
\usepackage{stmaryrd}
\usepackage{amsmath, amssymb}
\usepackage{systeme}
\usepackage{textcomp}

\begin{document}
\date{Septiembre de 2016}
\title{Tramos Revestidos\\Trabajo Práctico}
\author{Manuel F. Martín}

\maketitle


\section*{Tramo 1}


\begin{equation*}
  h = 1.42 m
 \qquad
  B_{f} = 5 m
 \qquad
  m = 1
 \qquad
  V_{max} = 5 m/s
\end{equation*}

\begin{equation*}
  n = 0.016 m^3/s
 \qquad
  Q_{t} = 11.63
 \qquad
  I = 0.00785
\end{equation*}

\begin{equation*}
 Q = \frac{1}{n} R^{2/3} I^{1/2} A
 \qquad
 R = \frac{A}{P}
\end{equation*}

\begin{equation*}
  \begin{cases}
    \dfrac{Q n}{I^{1/2}} = R^{2/3} A = \dfrac{A^{5/3}}{P^{2/3}} \\
    A = B_{f}y + m y^{2} \\
    P = B_{f} + 2 y \sqrt{1 + m^{2}}
  \end{cases}
\end{equation*}
\begin{equation*}
  \dfrac{Q n}{I^{1/2}} = \dfrac{A^{5/3}}{P^{2/3}} = \sqrt[3]{\dfrac{A^{5}}{P^{2}}}
  \Rightarrow
  \dfrac{Q n}{I^{1/2}}^{3} = \dfrac{A^{5}}{P^{2}}
\end{equation*}

\begin{equation*}
  \dfrac{A^{5}}{P^{2}} = \dfrac{(5 y + y^{2})^{5}}{(5 + 2 \sqrt{2} y)^{2}} = 9.19
  \Rightarrow
  y_{1} = -5.7157 \vee y_{2} = 0.5955
\end{equation*}


Como el valor de $y$ no puede ser negativo, consideramos el valor de $y_{2}$

\begin{equation*}
  \begin{cases}
    A = 5 y - y^{2} = 3.33 m^{2} \\
    P = 5 + 2 \sqrt{2} y = 6.68 m
  \end{cases}
  \Rightarrow
  R = 0.50 m
  \Rightarrow
  \tau = \gamma R \, S \, 0.97 = 3.81 kg/m^{2}
\end{equation*}


\subsection*{Verificación}

\subsubsection*{Escurrimiento}

\begin{itemize}
 \item De la ecuación de continuidad obtenemos la velocidad media:
    \begin{equation*}
    V_{m} = \dfrac{Q}{A} = \dfrac{11.6}{3.93} = 2.95 m/s
    \end{equation*}
 \item Para obtener la $V_{min}$, sabiendo el tirante y el tipo de agua, poco limoso de limo fino, buscamos en
 la tabla de velocidades mínimas y obtenemos $V_{min} = 0.36 m/s$.
    \begin{equation*}
      \begin{cases}
      V_{min} = 0.36 m/s \\
      V_{max} = 5 m/s
      \end{cases}
      \Longrightarrow \quad
      V_{m} = 2.95 m/s
      \qquad
      VERIFICA
    \end{equation*}
\end{itemize}

\subsubsection*{Desborde}

\begin{equation*}
 r = 20\%
 \qquad
 h_{disp} = 1.42 m
\end{equation*}
\begin{align*}
 y + r &\leq h_{disp} \\
 0.5955 m + 0.2 \times 0.5955 m &\leq h_{disp} \\
 0.7146 m &\leq 1.42 m
 \qquad
 VERIFICA
\end{align*}



\section*{Tramo 2}


\begin{equation*}
  h = 1.82 m (progresiva 0.1 km)
 \qquad
  B_{f} = 5 m
 \qquad
  m = 1
 \qquad
  V_{max} = 5 m/s
\end{equation*}

\begin{equation*}
  n = 0.016 m^3/s
 \qquad
  Q_{t} = 11.6 m^{3}
 \qquad
  I = 0.00113
\end{equation*}

\begin{equation*}
 Q = \frac{1}{n} R^{2/3} I^{1/2} A
 \qquad
 R = \frac{A}{P}
\end{equation*}

\begin{equation*}
  \begin{cases}
    \dfrac{Q n}{I^{1/2}} = R^{2/3} A = \dfrac{A^{5/3}}{P^{2/3}} \\
    A = B_{f}y + m y^{2} \\
    P = B_{f} + 2 y \sqrt{1 + m^{2}}
  \end{cases}
\end{equation*}
\begin{equation*}
  \dfrac{Q n}{I^{1/2}} = \dfrac{A^{5/3}}{P^{2/3}} = \sqrt[3]{\dfrac{A^{5}}{P^{2}}}
  \Rightarrow
  \dfrac{Q n}{I^{1/2}}^{3} = \dfrac{A^{5}}{P^{2}}
\end{equation*}

\begin{equation*}
  \dfrac{A^{5}}{P^{2}} = \dfrac{(5 y + y^{2})^{5}}{(5 + 2 \sqrt{2} y)^{2}} = 168.31 m
  \Rightarrow
  y_{1} = -6.2333 m \vee y_{2} = 1.0568 m
\end{equation*}


Como el valor de $y$ no puede ser negativo, consideramos el valor de $y_{2}$

\begin{equation*}
  \begin{cases}
    A = 5 y - y^{2} = 6.40 m^{2} \\
    P = 5 + 2 \sqrt{2} y = 7.99 m
  \end{cases}
  \Rightarrow
  R = 0.80 m
  \Rightarrow
  \tau = \gamma R \, S \, 0.97 = 0.88 kg/m^{2}
\end{equation*}


\subsection*{Verificación}

\subsubsection*{Escurrimiento}

\begin{itemize}
 \item De la ecuación de continuidad obtenemos la velocidad media:
    \begin{equation*}
    V_{m} = \dfrac{Q}{A} = \dfrac{11.6}{6.4} = 1.81 m/s
    \end{equation*}
 \item Para obtener la $V_{min}$, sabiendo el tirante y el tipo de agua, poco limoso de limo fino, buscamos en
 la tabla de velocidades mínimas y obtenemos $V_{min} = 0.47 m/s$.
    \begin{equation*}
      \begin{cases}
      V_{min} = 0.47 m/s \\
      V_{max} = 5 m/s
      \end{cases}
      \Longrightarrow \quad
      V_{m} = 1.81 m/s
      \qquad
      VERIFICA
    \end{equation*}
\end{itemize}

\subsubsection*{Desborde}

\begin{equation*}
 r = 20\%
 \qquad
 h_{disp} = 1.82 m
\end{equation*}
\begin{align*}
 y + r &\leq h_{disp} \\
 1.0568 m + 0.2 \times 1.0568 m &\leq h_{disp} \\
 1.27 m &\leq 1.82 m
 \qquad
 VERIFICA
\end{align*}


\section*{Conducto Circular}

\subsection*{Dos Conductos}

\begin{equation*}
  D = 1 m
 \qquad
  I = 7 \textperthousand = 0.007
 \qquad
  n = 0.016 m^3/s
\end{equation*}

\begin{equation*}
  Q_{1} = 2.75 m/s^{2}
 \qquad
  Q_{PC} = \dfrac{Q_{1}}{2} = 1.375 m^{3}/s
\end{equation*}

\begin{equation*}
  \begin{cases}
    Q = V A \\
    V = \dfrac{1}{n} R^{2/3} I^{1/2}
  \end{cases}
  \Longrightarrow \quad
  Q = \dfrac{1}{n} R^{2/3} I^{1/2} A = \dfrac{1}{0.016} (\dfrac{1}{4})^{2/3} 0.007^{1/2} \dfrac{\pi D^{2}}{4} = 1.63 m^{3}/s
\end{equation*}

\begin{equation*}
  R = \dfrac{A}{P} = \dfrac{\pi D^{2}}{4} \dfrac{1}{\pi D} = \dfrac{D}{4} = \dfrac{1}{4}
\end{equation*}

\begin{equation*}
  \dfrac{Q}{Q_{0}} = \dfrac{1.375}{1.63} = 0.84
  \Longrightarrow \quad
  y = 0.7 m
\end{equation*}

\begin{equation*}
  \begin{cases}
    \dfrac{V}{V_{0}} = 1.13 \\
    V_{0} = \dfrac{Q}{A} = \dfrac{1.375}{\pi D^{2}} 4 = 1.75 m/s
  \end{cases}
  \Longrightarrow \quad
  V = 1.13 1.75 = 1.98 < V_{max} = 5 m/s
  \qquad
  VERIFICA
\end{equation*}



\subsection*{Verificación}

\begin{equation*}
  \dfrac{Q_{m}}{I^{1/2}} = 0.263
\end{equation*}

Propongo: $y = 0.7 m$

\begin{equation*}
  \theta = 2 \cos{1 - \dfrac{2 y}{D}}^{-1} = 3.965
\end{equation*}

\begin{equation*}
  \begin{cases}
    A = \dfrac{1}{8} (\theta - \sen{\theta}) D^{2} = 0.587 m^{2} \\
    P = \dfrac{1}{2} \theta D = 1.9825 m
  \end{cases}
  \Longrightarrow \quad
  R = 0.296 m
\end{equation*}

\begin{equation*}
  A R^{2/3} = 0.261
\end{equation*}

\begin{equation*}
  \dfrac{Q_{n}}{I^{1/2}} \cong A R^{2/3}
\end{equation*}

$\therefore ES\ CORRECTO$




\end{document}
