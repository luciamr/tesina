\documentclass[10.5pt]{article}
\usepackage{a4wide}
%\usepackage[cm]{fullpage} %si lo queremos mas ancho

\usepackage[spanish, english, activeacute]{babel}
\usepackage[utf8]{inputenc}
\usepackage{enumerate}
\usepackage{listings}
\usepackage{stmaryrd}
\usepackage{amsmath, amssymb}
\usepackage{systeme}
\usepackage{textcomp}
\usepackage{tikz}
\usetikzlibrary{shapes,arrows, decorations.markings, positioning}


\begin{document}
\date{14 de Septiembre de 2016}
\title{Hidráulica de Canales Abiertos\\Trabajo Práctico Integrador}
\author{Manuel Álvarez\\Manuel F. Martín}

\maketitle


\section*{Etapa 1}

\begin{enumerate}[a)]
 \item \underline{\bf Rasante}
 
 Para trazar la rasante tuvimos en cuenta las siguientes pautas:
  \begin{itemize}
   \item Realizar exclusivamente excavaciones, lo que implica que la rasante se ubique por debajo de la cota del fondo existente o la supere
   en a lo sumo $20 cm$ en $200 m$, ya que realizar un relleno es muy dificil y costoso.
   \item Intentar que la misma se adapte de la mejor manera posible a la topografía existente, de manera tal que se realice la menor cantidad
   de excavaciones posibles.
  \end{itemize}
  
 \item \underline{\bf Dimensionamiento de los tramos de material erosionable}
  
  La revancha que adoptamos es de un $20\%$. Si bien el valor de ésta no se calcula con un fórmula matemática, podemos tener en cuenta que 
  en general, para canales de débil pendiente, se utilizan revanchas que oscilan entre el $5\%$ y el $30\%$. Para elegir el valor de la nuestra
  consideramos la presencia de un camino y un alambrado cercanos.
    
  \underline{\bf Método de la velocidad máxima}
  
  \underline{\bf Tramo 1}

\begin{equation*}
  r = 20\%
 \qquad
  h_{disp} = 1.12 m \,\text{(en progresiva }5.6 km \text{)}
 \qquad
  I = 0.00147
\end{equation*}

\begin{equation*}
  Q = 3.85 m^3/s
 \qquad
  V_{max} = 1.5 m/s
 \qquad
  n = 0.035
 \qquad
  m = 1
\end{equation*}

\begin{equation*}
  \text{Relación de huecos: 0.3}
  \qquad
  \text{Agua: poco limosa, limo muy fino}
\end{equation*}

Calculamos el área estable

\begin{equation*}
 A_{e} = \dfrac{Q}{V_{max}} = \dfrac{3.85 m^3/s}{1.5 m/s} = 2.57 m^2
\end{equation*}

Luego, aplicando la ecuación de Chazy Manning
\begin{equation*}
 V_{max} =  \frac{1}{n} R^{2/3} I^{1/2}
 \Longrightarrow \quad
 R = 1.6 m
\end{equation*}

\begin{equation*}
 P = \dfrac{A_{e}}{R} = \dfrac{2.57 m^2}{1.6 m} = 1.61 m
\end{equation*}

\begin{equation*}
  \begin{cases}
    A_e = B_{f}y + m y^{2} \\
    P = B_{f} + 2 y \sqrt{1 + m^{2}}
  \end{cases}
  \Longrightarrow \quad
  \text{sólo soluciones complejas}
\end{equation*}

  \underline{\bf Tramo 2}

\begin{equation*}
  r = 20\%
 \qquad
  h_{disp} = 1.83 m \,\text{(en progresiva }4.1 km \text{)}
 \qquad
  I = 0.00105
\end{equation*}

\begin{equation*}
  Q = 6.6 m^3/s
 \qquad
  V_{max} = 1.5 m/s
 \qquad
  n = 0.035
 \qquad
  m = 1
\end{equation*}

\begin{equation*}
  \text{Relación de huecos: 0.3}
  \qquad
  \text{Agua: poco limosa, limo muy fino}
\end{equation*}

Calculamos el área estable

\begin{equation*}
 A_{e} = \dfrac{Q}{V_{max}} = \dfrac{6.6 m^3/s}{1.5 m/s} = 4.4 m^2
\end{equation*}

Luego, aplicando la ecuación de Chazy Manning
\begin{equation*}
 V_{max} =  \frac{1}{n} R^{2/3} I^{1/2}
 \Longrightarrow \quad
 R = 2.06 m
\end{equation*}

\begin{equation*}
 P = \dfrac{A_{e}}{R} = \dfrac{4.4 m^2}{2.06 m} = 2.14 m
\end{equation*}

\begin{equation*}
  \begin{cases}
    A_e = B_{f}y + m y^{2} \\
    P = B_{f} + 2 y \sqrt{1 + m^{2}}
  \end{cases}
  \Longrightarrow \quad
  \text{sólo soluciones complejas}
\end{equation*}

  \underline{\bf Tramo 3}

\begin{equation*}
  r = 20\%
 \qquad
  h_{disp} = 2.47 m \,\text{(en progresiva }2.5 km \text{)}
 \qquad
  I = 0.00086
\end{equation*}

\begin{equation*}
  Q = 11.6 m^3/s
 \qquad
  V_{max} = 1.5 m/s
 \qquad
  n = 0.035
 \qquad
  m = 1
\end{equation*}

\begin{equation*}
  \text{Relacion de huecos: 0.3}
  \qquad
  \text{Agua: poco limosa, limo muy fino}
\end{equation*}

Calculamos el área estable

\begin{equation*}
 A_{e} = \dfrac{Q}{V_{max}} = \dfrac{11.6 m^3/s}{1.5 m/s} = 7.73 m^2
\end{equation*}

Luego, aplicando la ecuación de Chazy Manning
\begin{equation*}
 V_{max} =  \frac{1}{n} R^{2/3} I^{1/2}
 \Longrightarrow \quad
 R = 2.4 m
\end{equation*}

\begin{equation*}
 P = \dfrac{A_{e}}{R} = \dfrac{7.73 m^2}{2.4 m} = 3.22 m
\end{equation*}

\begin{equation*}
  \begin{cases}
    A_e = B_{f}y + m y^{2} \\
    P = B_{f} + 2 y \sqrt{1 + m^{2}}
  \end{cases}
  \Longrightarrow \quad
  \text{sólo soluciones complejas}
\end{equation*}

  
  \underline{Conclusión}
  
  Este método no es aplicable para estos casos en particular dados los datos disponibles.
  Para poder obtener soluciones reales podríamos bajar la $V_{max}$, sin embargo dejaría de ser el método de la velocidad máxima.
  Debido a ésto debemos utilizar el método de la fuerza tractiva para poder hacer el dimensionamiento en forma más precisa.
  
  \underline{\bf Método de la fuerza tractiva}
  
  \underline{\bf Tramo 1}

\begin{equation*}
  r = 20\%
 \qquad
  h_{disp} = 1.12 m \,\text{(en progresiva }5.6 km \text{)}
 \qquad
  I = 0.00147
\end{equation*}

\begin{equation*}
  Q = 3.85 m^3/s
 \qquad
  V_{max} = 1.5 m/s
 \qquad
  n = 0.035
 \qquad
  m = 1
\end{equation*}

\begin{equation*}
  \text{Relación de huecos: 0.3}
  \qquad
  \text{Agua: poco limosa, limo muy fino}
\end{equation*}

Suelo cohesivo:
\begin{align*}
 \tau_{resistente\,talud} &= \tau_{resistente\,fondo} \\
 \tau_{act\,talud} = 0.75 \, \gamma \, y \, I &\leq \tau_{resistente\,talud} \\
 \tau_{act\,fondo} = 0.97 \, \gamma \, y \, I &\leq \tau_{resistente\,fondo} \Longrightarrow \text{más condicionante}
\end{align*}

Con la relación de huecos y el tipo de suelo, buscamos en la tabla que vincula fuerza tractiva con relación de huecos y obtenemos
el $\tau_{resistente fondo} = 0.38 lb/ ft^2 = 1.855 kg/m^2$.

\begin{align*}
 y &\leq \dfrac{\tau_{resistente\,fondo}}{0.97 \, \gamma \, I} \\
 y &\leq 1.30 m
\end{align*}

A pesar de que el suelo podría soportar un tirante mayor a nuestra $h_{disp}$, en este caso la $h_{disp}$ sigue siendo $1.12 m$, ya que si adoptáramos 
un $h_{disp}$ mayor, el canal podría desbordarse.
Considerando la revancha obtenemos $1.2 y = 1.12 m$, luego $y = 0.93 m$.


\begin{equation*}
  \begin{cases}
    Q = V A \\
    V =  \frac{1}{n} R^{2/3} I^{1/2}
  \end{cases}
  \Longrightarrow \quad
  Q = \frac{1}{n} R^{2/3} I^{1/2} A
\end{equation*}


\begin{equation*}
  \begin{cases}
    Q = \frac{1}{n} R^{2/3} I^{1/2} A \\
    R = \dfrac{A}{P} \\
    A = B_{f}y + m y^{2} \\
    P = B_{f} + 2 y \sqrt{1 + m^{2}}
  \end{cases}
  \Longrightarrow \quad
  \begin{cases}
    Q = \frac{1}{0.035} R^{2/3} 0.00147^{1/2} A \\
    R = \dfrac{A}{P} \\
    A = B_{f}0.93 m + (0.93 m)^2 \\
    P = B_{f} + 2 \, 0.93 m \sqrt{2}
  \end{cases}
  \Longrightarrow \quad
  B_{f} = 3.92 m
\end{equation*}

Debido a que la apreciación debe ser de $0.1 m$, adoptamos $B_{f} = 4 m$.
Luego, calculamos el tirante correspondiente para el $B_{f}$ adoptado.

\begin{equation*}
  \begin{cases}
    Q = \frac{1}{n} R^{2/3} I^{1/2} A \\
    R = \dfrac{A}{P} \\
    A = B_{f}y + m y^{2} \\
    P = B_{f} + 2 y \sqrt{1 + m^{2}}
  \end{cases}
  \Longrightarrow \quad
  \begin{cases}
    Q = \frac{1}{0.035} R^{2/3} 0.00147^{1/2} A \\
    R = \dfrac{A}{P} \\
    A = 4 m y + y^2 \\
    P = 4 m + 2 y \sqrt{2}
  \end{cases}
  \Longrightarrow \quad
  y = 0.92 m
\end{equation*}

Luego,
\begin{equation*}
  \begin{cases}
    A = B_{f}y + m y^{2} \\
    P = B_{f} + 2 y \sqrt{1 + m^{2}}
  \end{cases}
  \Longrightarrow
  \begin{cases}
    A = 4.53 m^2 \\
    P = 6.60 m
  \end{cases}
  \Longrightarrow \quad
  R = 0.69 m
\end{equation*}


\underline{Verificación}

\begin{itemize}
 \item Tensiones
    \begin{align*}
    0.97 \, \gamma \, R \, S &\leq \tau_{resistente\,fondo} \\
    0.97 \, 1000 kg/m^3 \, 0.69 m \, 0.00147 &\leq \tau_{resistente\,fondo} \\
    0.984 kg/m^2 &\leq 1.855 kg/m^2
    \qquad
    \therefore VERIFICA 
    \end{align*}

 \item Escurrimiento
    \begin{itemize}
    \item De la ecuación de continuidad obtenemos la velocidad media:
	\begin{equation*}
	V_{m} = \dfrac{Q}{A} = \dfrac{3.85 m^3/s}{4.53 m^2} = 0.85 m/s
	\end{equation*}
    \item Para obtener la $V_{min}$, sabiendo el tirante y el tipo de agua, poco limoso de limo fino, buscamos en
    la tabla de velocidades mínimas y obtenemos $V_{min} = 0.44 m/s$.
	\begin{equation*}
	  \begin{cases}
	  V_{min} = 0.44 m/s \\
	  V_{max} = 1.5 m/s
	  \end{cases}
	  \Longrightarrow \quad
	  V_{m} = 0.85 m/s
	  \qquad
	  \therefore VERIFICA
	\end{equation*}
    \end{itemize}
 
 \item{Desborde}
    \begin{equation*}
    r = 20\%
    \qquad
    h_{disp} = 1.12 m
    \end{equation*}
    \begin{align*}
    y + r &\leq h_{disp} \\
    0.92 m + 0.2 \times 0.92 m &\leq h_{disp} \\
    1.10 m &\leq 1.12 m
    \qquad
    \therefore VERIFICA
    \end{align*}
\end{itemize}

  \underline{\bf Tramo 2}

\begin{equation*}
  r = 20\%
 \qquad
  h_{disp} = 1.83 m \,\text{(en progresiva }4.1 km \text{)}
 \qquad
  I = 0.00105
\end{equation*}

\begin{equation*}
  Q = 6.6 m^3/s
 \qquad
  V_{max} = 1.5 m/s
 \qquad
  n = 0.035
 \qquad
  m = 1
\end{equation*}

\begin{equation*}
  \text{Relación de huecos: 0.3}
  \qquad
  \text{Agua: poco limosa, limo muy fino}
\end{equation*}

Suelo cohesivo:
\begin{align*}
 \tau_{resistente\,talud} &= \tau_{resistente\,fondo} \\
 \tau_{act\,talud} = 0.75 \, \gamma \, y \, I &\leq \tau_{resistente\,talud} \\
 \tau_{act\,fondo} = 0.97 \, \gamma \, y \, I &\leq \tau_{resistente\,fondo} \Longrightarrow \text{más condicionante}
\end{align*}

Con la relación de huecos y el tipo de suelo, buscamos en la tabla que vincula fuerza tractiva con relación de huecos y obtenemos
el $\tau_{resistente\,fondo} = 0.38 lb/ft^2 = 1.855 kg/m^2$

\begin{align*}
 y &\leq \dfrac{\tau_{resistente\,fondo}}{0.97 \, \gamma \, I} \\
 y &\leq 1.82 m
\end{align*}

A pesar de que la altura máxima disponible es de $1.83 m$, las características del suelo nos limitan la $h_{disp}$ a $1.82 m$.
Considerando la revancha obtenemos $1.2 y = 1.82 m$, luego $y = 1.52 m$.

\begin{equation*}
  \begin{cases}
    Q = V \, A \\
    V =  \frac{1}{n} R^{2/3} I^{1/2}
  \end{cases}
  \Longrightarrow \quad
  Q = \frac{1}{n} R^{2/3} I^{1/2} A
\end{equation*}


\begin{equation*}
  \begin{cases}
    Q = \frac{1}{n} R^{2/3} I^{1/2} A \\
    R = \dfrac{A}{P} \\
    A = B_{f}y + m y^{2} \\
    P = B_{f} + 2 y \sqrt{1 + m^{2}}
  \end{cases}
  \Longrightarrow \quad
  \begin{cases}
    Q = \frac{1}{0.035} R^{2/3} 0.00105^{1/2} A \\
    R = \dfrac{A}{P} \\
    A = B_{f}1.52 m + (1.52 m)^2 \\
    P = B_{f} + 2 \, 1.52 m \sqrt{2}
  \end{cases}
  \Longrightarrow \quad
  B_{f} = 3.289 m
\end{equation*}

Ya que no se debe disminuir la sección aguas abajo, adoptamos $B_{f} = 4 m$, de modo que sea igual al utilizado en el tramo anterior.
Luego, calculamos el tirante correspondiente para el $B_{f}$ adoptado.

\begin{equation*}
  \begin{cases}
    Q = \frac{1}{n} R^{2/3} I^{1/2} A \\
    R = \dfrac{A}{P} \\
    A = B_{f}y + m y^{2} \\
    P = B_{f} + 2 y \sqrt{1 + m^{2}}
  \end{cases}
  \Longrightarrow \quad
  \begin{cases}
    Q = \frac{1}{0.035} R^{2/3} 0.00105^{1/2} A \\
    R = \dfrac{A}{P} \\
    A = 4 m y + y^2 \\
    P = 4 m + 2 y \sqrt{2}
  \end{cases}
  \Longrightarrow \quad
  y = 1.38 m
\end{equation*}

Luego,
\begin{equation*}
  \begin{cases}
    A = B_{f}y + m y^{2} \\
    P = B_{f} + 2 y \sqrt{1 + m^{2}}
  \end{cases}
  \Longrightarrow
  \begin{cases}
    A = 7.42 m^2 \\
    P = 7.90 m
  \end{cases}
  \Longrightarrow \quad
  R = 0.94 m
\end{equation*}


\underline{Verificación}

\begin{itemize}
 \item Tensiones
    \begin{align*}
    0.97 \, \gamma \, R \, S &\leq \tau_{resistente\,fondo} \\
    0.97 \, 1000 kg/m^3 \, 0.94 m \, 0.00105 &\leq \tau_{resistente\,fondo} \\
    0.957 kg/m^2 &\leq 1.855 kg/m^2
    \qquad
    \therefore VERIFICA 
    \end{align*}

  \item Escurrimiento
    \begin{itemize}
    \item De la ecuación de continuidad obtenemos la velocidad media:
	\begin{equation*}
	V_{m} = \dfrac{Q}{A} = \dfrac{6.6 m^3/s}{7.42 m^2} = 0.89 m/s
	\end{equation*}
    \item Para obtener la $V_{min}$, sabiendo el tirante y el tipo de agua, poco limoso de limo fino, buscamos en
    la tabla de velocidades mínimas y obtenemos $V_{min} = 0.53 m/s$.
	\begin{equation*}
	  \begin{cases}
	  V_{min} = 0.53 m/s \\
	  V_{max} = 1.5 m/s
	  \end{cases}
	  \Longrightarrow \quad
	  V_{m} = 0.89 m/s
	  \qquad
	  \therefore VERIFICA
	\end{equation*}
    \end{itemize}

 \item Desborde
    \begin{equation*}
    r = 20\%
    \qquad
    h_{disp} = 1.82 m
    \end{equation*}
    \begin{align*}
    y + r &\leq h_{disp} \\
    1.38 m + 0.2 \times 1.38 m &\leq h_{disp} \\
    1.66 m &\leq 1.82 m
    \qquad
    \therefore VERIFICA
    \end{align*}
\end{itemize}


  \underline{\bf Tramo 3}

\begin{equation*}
  r = 20\%
 \qquad
  h_{disp} = 2.47 m \,\text{(en progresiva }2.5 km \text{)}
 \qquad
  I = 0.00086
\end{equation*}

\begin{equation*}
  Q = 11.6 m^3/s
 \qquad
  V_{max} = 1.5 m/s
 \qquad
  n = 0.035
 \qquad
  m = 1
\end{equation*}

\begin{equation*}
  \text{Relacion de huecos: 0.3}
  \qquad
  \text{Agua: poco limosa, limo muy fino}
\end{equation*}

Suelo cohesivo:
\begin{align*}
 \tau_{resistente\,talud} &= \tau_{resistente\,fondo} \\
 \tau_{act\,talud} = 0.75 \, \gamma \, y \, I &\leq \tau_{resistente\,talud} \\
 \tau_{act\,fondo} = 0.97 \, \gamma \, y \, I &\leq \tau_{resistente\,fondo} \Longrightarrow \text{más condicionante}
\end{align*}

Con la relación de huecos y el tipo de suelo, buscamos en la tabla que vincula fuerza tractiva con relación de huecos y obtenemos
el $\tau_{resistente\,fondo} = 0.38 lb/ft^2 = 1.855 kg/m^2$

\begin{align*}
 y &\leq \dfrac{\tau_{resistente\,fondo}}{0.97 \, \gamma \, I} \\
 y &\leq 2.21 m
\end{align*}

A pesar de que la altura máxima disponible es de $2.47 m$, las características del suelo nos limitan la $h_{disp}$ a $2.21 m$.
Considerando la revancha obtenemos $1.2 y = 2.21 m$, luego $y = 1.84 m$.

\begin{equation*}
  \begin{cases}
    Q = V \, A \\
    V =  \frac{1}{n} R^{2/3} I^{1/2}
  \end{cases}
  \Longrightarrow \quad
  Q = \frac{1}{n} R^{2/3} I^{1/2} A
\end{equation*}

\begin{equation*}
  \begin{cases}
    Q = \frac{1}{n} R^{2/3} I^{1/2} A \\
    R = \dfrac{A}{P} \\
    A = B_{f}y + m y^{2} \\
    P = B_{f} + 2 y \sqrt{1 + m^{2}}
  \end{cases}
  \Longrightarrow \quad
  \begin{cases}
    Q = \frac{1}{0.035} R^{2/3} 0.00086^{1/2} A \\
    R = \dfrac{A}{P} \\
    A = B_{f}1.84 m + (1.84 m)^2 \\
    P = B_{f} + 2 \, 1.84 m \sqrt{2}
  \end{cases}
  \Longrightarrow \quad
  B_{f} = 4.75 m
\end{equation*}

Adoptamos $B_{f} = 5 m$, debido a que debe haber una diferencia de al menos $0.5 m$ entre las secciones, ya que de no existir resulta
muy dificil la ejecución de las diferencias por medio de la maquinaria con que se va a llevar a cabo el trabajo.
Luego, calculamos el tirante correspondiente para el $B_{f}$ adoptado.

\begin{equation*}
  \begin{cases}
    Q = \frac{1}{n} R^{2/3} I^{1/2} A \\
    R = \dfrac{A}{P} \\
    A = B_{f}y + m y^{2} \\
    P = B_{f} + 2 y \sqrt{1 + m^{2}}
  \end{cases}
  \Longrightarrow \quad
  \begin{cases}
    Q = \frac{1}{0.035} R^{2/3} 0.00086^{1/2} A \\
    R = \dfrac{A}{P} \\
    A = 5 m y + y^2 \\
    P = 5 m + 2 y \sqrt{2}
  \end{cases}
  \Longrightarrow \quad
  y = 1.79 m
\end{equation*}

Luego,
\begin{equation*}
  \begin{cases}
    A = B_{f}y + m y^{2} \\
    P = B_{f} + 2 y \sqrt{1 + m^{2}}
  \end{cases}
  \Longrightarrow
  \begin{cases}
    A = 12.15 m^2 \\
    P = 10.06 m
  \end{cases}
  \Longrightarrow \quad
  R = 1.21 m
\end{equation*}


\underline{Verificación}

\begin{itemize}
 \item Tensiones
    \begin{align*}
    0.97 \, \gamma \, R \, S &\leq \tau_{resistente\,fondo} \\
    0.97 \, 1000 kg/m^3 \, 1.22 m \, 0.00086 &\leq \tau_{resistente\,fondo} \\
    1.018 kg/m^2 &\leq 1.855 kg/m^2
    \qquad
    \therefore VERIFICA 
    \end{align*}

 \item Escurrimiento
    \begin{itemize}
    \item De la ecuación de continuidad obtenemos la velocidad media:
	\begin{equation*}
	V_{m} = \dfrac{Q}{A} = \dfrac{11.6 m^3/s}{12.13 m^2} = 0.95 m/s
	\end{equation*}
    \item Para obtener la $V_{min}$, sabiendo el tirante y el tipo de agua, poco limoso de limo fino, buscamos en
    la tabla de velocidades mínimas y obtenemos $V_{min} = 0.63 m/s$.
	\begin{equation*}
	  \begin{cases}
	  V_{min} = 0.63 m/s \\
	  V_{max} = 1.5 m/s
	  \end{cases}
	  \Longrightarrow \quad
	  V_{m} = 0.95 m/s
	  \qquad
	  \therefore VERIFICA
	\end{equation*}
    \end{itemize}

 \item Desborde
    \begin{equation*}
    r = 20\%
    \qquad
    h_{disp} = 2.21 m
    \end{equation*}
    \begin{align*}
    y + r &\leq h_{disp} \\
    1.79 m + 0.2 \times 1.79 m &\leq h_{disp} \\
    2.15 m &\leq 2.21 m
    \qquad
    \therefore VERIFICA
    \end{align*}
\end{itemize}


 
 \item \underline{\bf Diagrama de flujo del método de la fuerza tractiva}
   
   %\input{diagrama.tex}
   
 \item \underline{\bf Determinación de tirantes en los tramos revestidos}

   \section*{Tramo 1r}


\begin{equation*}
  h = 1.42 m
 \qquad
  B_{f} = 5 m
 \qquad
  m = 1
 \qquad
  V_{max} = 5 m/s
\end{equation*}

\begin{equation*}
  n = 0.016 m^3/s
 \qquad
  Q_{t} = 11.63
 \qquad
  I = 0.00785
\end{equation*}

\begin{equation*}
 Q = \frac{1}{n} R^{2/3} I^{1/2} A
 \qquad
 R = \frac{A}{P}
\end{equation*}

\begin{equation*}
  \begin{cases}
    \dfrac{Q n}{I^{1/2}} = R^{2/3} A = \dfrac{A^{5/3}}{P^{2/3}} \\
    A = B_{f}y + m y^{2} \\
    P = B_{f} + 2 y \sqrt{1 + m^{2}}
  \end{cases}
\end{equation*}

\begin{equation*}
  \dfrac{Q n}{I^{1/2}} = \dfrac{A^{5/3}}{P^{2/3}} = \sqrt[3]{\dfrac{A^{5}}{P^{2}}}
  \Longrightarrow
  \dfrac{Q n}{I^{1/2}}^{3} = \dfrac{A^{5}}{P^{2}}
\end{equation*}

\begin{equation*}
  \dfrac{A^{5}}{P^{2}} = \dfrac{(5 y + y^{2})^{5}}{(5 + 2 \sqrt{2} y)^{2}} = 9.19
  \Longrightarrow
  y_{1} = -5.7157 \vee y_{2} = 0.5955
\end{equation*}

Como el valor de $y$ no puede ser negativo, consideramos el valor de $y_{2}$

\begin{equation*}
  \begin{cases}
    A = 5 y - y^{2} = 3.33 m^{2} \\
    P = 5 + 2 \sqrt{2} y = 6.68 m
  \end{cases}
  \Longrightarrow
  R = 0.50 m
  \Longrightarrow
  \tau = \gamma R \, S \, 0.97 = 3.81 kg/m^{2}
\end{equation*}


\subsection*{Verificación}

\subsubsection*{Escurrimiento}

\begin{itemize}
 \item De la ecuación de continuidad obtenemos la velocidad media:
    \begin{equation*}
    V_{m} = \dfrac{Q}{A} = \dfrac{11.6}{3.93} = 2.95 m/s
    \end{equation*}
 \item Para obtener la $V_{min}$, sabiendo el tirante y el tipo de agua, poco limoso de limo fino, buscamos en
 la tabla de velocidades mínimas y obtenemos $V_{min} = 0.36 m/s$.
    \begin{equation*}
      \begin{cases}
      V_{min} = 0.36 m/s \\
      V_{max} = 5 m/s
      \end{cases}
      \Longrightarrow \quad
      V_{m} = 2.95 m/s
      \qquad
      \therefore VERIFICA
    \end{equation*}
\end{itemize}

\subsubsection*{Desborde}

\begin{equation*}
 r = 20\%
 \qquad
 h_{disp} = 1.42 m
\end{equation*}
\begin{align*}
 y + r &\leq h_{disp} \\
 0.5955 m + 0.2 \times 0.5955 m &\leq h_{disp} \\
 0.7146 m &\leq 1.42 m
 \qquad
 \therefore VERIFICA
\end{align*}


   \underline{\bf Tramo 2r}


\begin{equation*}
  r = 20\%
  \quad
  h_{disp} = 1.82 m (progresiva 0.1 km)
 \qquad
  I = 0.00113
 \qquad
  B_{f} = 5 m
\end{equation*}

\begin{equation*}
  Q_{t} = 11.6 m^{3}
 \qquad
  V_{max} = 5 m/s
 \qquad
  n = 0.016 m^3/s
 \qquad
  m = 1
\end{equation*}

\begin{equation*}
 Q = \frac{1}{n} R^{2/3} I^{1/2} A
 \qquad
 R = \frac{A}{P}
\end{equation*}

\begin{equation*}
  \begin{cases}
    \dfrac{Q n}{I^{1/2}} = R^{2/3} A = \dfrac{A^{5/3}}{P^{2/3}} \\
    A = B_{f}y + m y^{2} \\
    P = B_{f} + 2 y \sqrt{1 + m^{2}}
  \end{cases}
  \Longrightarrow
  (\dfrac{Q n}{I^{1/2}})^{3} = \dfrac{A^{5}}{P^{2}}
\end{equation*}

Luego,
\begin{equation*}
  \dfrac{A^{5}}{P^{2}} = \dfrac{(5 y + y^{2})^{5}}{(5 + 2 \sqrt{2} y)^{2}} = 168.31 m
  \quad
  \Longrightarrow
  \quad
  y_{1} = -6.2333 m \vee y_{2} = 1.0568 m
\end{equation*}


Como el valor de $y$ no puede ser negativo, consideramos el valor de $y_{2}$

\begin{equation*}
  \begin{cases}
    A = 5 y - y^{2} = 6.40 m^{2} \\
    P = 5 + 2 \sqrt{2} y = 7.99 m
  \end{cases}
  \quad
  \Longrightarrow
  \quad
  R = 0.80 m
  \quad
  \Longrightarrow
  \quad
  \tau = \gamma R \, S \, 0.97 = 0.88 kg/m^{2}
\end{equation*}


\underline{\bf Verificación}

\begin{itemize}
 \item Escurrimiento
    \begin{itemize}
    \item De la ecuación de continuidad obtenemos la velocidad media:
	\begin{equation*}
	V_{m} = \dfrac{Q}{A} = \dfrac{11.6}{6.4} = 1.81 m/s
	\end{equation*}
    \item Para obtener la $V_{min}$, sabiendo el tirante y el tipo de agua, poco limoso de limo fino, buscamos en
    la tabla de velocidades mínimas y obtenemos $V_{min} = 0.47 m/s$.
	\begin{equation*}
	  \begin{cases}
	  V_{min} = 0.47 m/s \\
	  V_{max} = 5 m/s
	  \end{cases}
	  \Longrightarrow \quad
	  V_{m} = 1.81 m/s
	  \qquad
	  \therefore
	  VERIFICA
	\end{equation*}
    \end{itemize}
 \item Desborde
    \begin{equation*}
    r = 20\%
    \qquad
    h_{disp} = 1.82 m
    \end{equation*}
    \begin{align*}
    y + r &\leq h_{disp} \\
    1.0568 m + 0.2 \times 1.0568 m &\leq h_{disp} \\
    1.27 m &\leq 1.82 m
    \qquad
    \therefore
    VERIFICA
    \end{align*}
\end{itemize}



   
 \item \underline{\bf Conductos circulares}
 
 Suponemos que el caudal $Q_1$ que se descarga en el canal, lo hace en partes iguale por ambos conductos.
 
 \begin{equation*}
  D = 1 m
 \qquad
 I = 0.007
 \qquad
  n = 0.016
\end{equation*}

Primero comprobamos si los conductos trabajan a sección llena o parcialmente llena.

\begin{equation*}
  Q_{1} = 2.75 m^3/s
 \qquad
  Q_{PC} = \dfrac{Q_{1}}{2} = 1.375 m^{3}/s
\end{equation*}

\begin{equation*}
  \begin{cases}
    Q = V \, A \\
    V = \dfrac{1}{n} R^{2/3} I^{1/2}
  \end{cases}
  \Longrightarrow \quad
  Q = \dfrac{1}{n} R^{2/3} I^{1/2} A = \dfrac{1}{0.016} (\dfrac{1}{4}m)^{2/3} 0.007^{1/2} \dfrac{\pi \, (1 m)^2}{4} = 1.63 m^{3}/s
\end{equation*}

De ésto se desprende que los conductos trabajan a sección parcialmente llena.

\begin{equation*}
  R = \dfrac{A}{P} = \dfrac{\pi \, (1 m)^{2}}{4} \dfrac{1}{\pi \, 1 m} = \dfrac{1}{4} m
\end{equation*}

A partir de la relación entre $Q_{PC}$ y $Q_0$ miramos la tabla de curvas adimensionales de elementos geométricos de una sección circular
y obtenemos el valor de $y$. Teniendo en cuenta el $y$ obtenido hallamos la relación entre $V$ y $V_0$ para luego determinar la velocidad media.

\begin{equation*}
  \dfrac{Q}{Q_{0}} = \dfrac{1.375}{1.63} = 0.84
  \quad
  \Longrightarrow \quad
  y = 0.7 m
\end{equation*}

\begin{equation*}
  \begin{cases}
    \dfrac{V}{V_{0}} = 1.13 \\
    V_{0} = \dfrac{Q}{A} = \dfrac{1.375}{\pi D^{2}} 4 = 1.75 m/s
  \end{cases}
  \Longrightarrow \quad
  V = 1.13 \, 1.75 m/s = 1.98 m/s < V_{max} = 5 m/s
  \qquad
  \therefore
  VERIFICA
\end{equation*}



\underline{Verificación}

Para llevar a cabo la verificación del resultado anteriormente obtenido, proponemos la resolución del mismo problema mediante otro método.

\begin{equation*}
  \dfrac{Q_{m}}{I^{1/2}} = 0.263
\end{equation*}

Propongo: $y = 0.7 m$

\begin{equation*}
  \theta = 2 \cos{1 - \dfrac{2 y}{D}}^{-1} = 3.965
\end{equation*}

\begin{equation*}
  \begin{cases}
    A = \dfrac{1}{8} (\theta - \sin{\theta}) D^{2} = 0.587 m^{2} \\
    P = \dfrac{1}{2} \theta D = 1.9825 m
  \end{cases}
  \Longrightarrow \quad
  R = 0.296 m
\end{equation*}

\begin{equation*}
  A R^{2/3} = 0.261
\end{equation*}

\begin{equation*}
  \dfrac{Q_{n}}{I^{1/2}} \cong A R^{2/3}
\end{equation*}

$\therefore ES\ CORRECTO$


\end{enumerate}


























 








\end{document}

