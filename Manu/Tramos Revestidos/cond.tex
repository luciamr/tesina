\begin{equation*}
  D = 1 m
 \qquad
 I = 0.007
 \qquad
  n = 0.016
\end{equation*}

Primero comprobamos si los conductos trabajan a sección llena o parcialmente llena.

\begin{equation*}
  Q_{1} = 2.75 m^3/s
 \qquad
  Q_{PC} = \dfrac{Q_{1}}{2} = 1.375 m^{3}/s
\end{equation*}

\begin{equation*}
  \begin{cases}
    Q = V \, A \\
    V = \dfrac{1}{n} R^{2/3} I^{1/2}
  \end{cases}
  \Longrightarrow \quad
  Q = \dfrac{1}{n} R^{2/3} I^{1/2} A = \dfrac{1}{0.016} (\dfrac{1}{4}m)^{2/3} 0.007^{1/2} \dfrac{\pi \, (1 m)^2}{4} = 1.63 m^{3}/s
\end{equation*}

De ésto se desprende que los conductos trabajan a sección parcialmente llena.

\begin{equation*}
  R = \dfrac{A}{P} = \dfrac{\pi \, (1 m)^{2}}{4} \dfrac{1}{\pi \, 1 m} = \dfrac{1}{4} m
\end{equation*}

A partir de la relación entre $Q_{PC}$ y $Q_0$ miramos la tabla de curvas adimensionales de elementos geométricos de una sección circular
y obtenemos el valor de $y$. Teniendo en cuenta el $y$ obtenido hallamos la relación entre $V$ y $V_0$ para luego determinar la velocidad media.

\begin{equation*}
  \dfrac{Q}{Q_{0}} = \dfrac{1.375}{1.63} = 0.84
  \quad
  \Longrightarrow \quad
  y = 0.7 m
\end{equation*}

\begin{equation*}
  \begin{cases}
    \dfrac{V}{V_{0}} = 1.13 \\
    V_{0} = \dfrac{Q}{A} = \dfrac{1.375}{\pi D^{2}} 4 = 1.75 m/s
  \end{cases}
  \Longrightarrow \quad
  V = 1.13 \, 1.75 m/s = 1.98 m/s < V_{max} = 5 m/s
  \qquad
  \therefore
  VERIFICA
\end{equation*}



\underline{Verificación}

Para llevar a cabo la verificación del resultado anteriormente obtenido, proponemos la resolución del mismo problema mediante otro método.

\begin{equation*}
  \dfrac{Q_{m}}{I^{1/2}} = 0.263
\end{equation*}

Propongo: $y = 0.7 m$

\begin{equation*}
  \theta = 2 \cos{1 - \dfrac{2 y}{D}}^{-1} = 3.965
\end{equation*}

\begin{equation*}
  \begin{cases}
    A = \dfrac{1}{8} (\theta - \sin{\theta}) D^{2} = 0.587 m^{2} \\
    P = \dfrac{1}{2} \theta D = 1.9825 m
  \end{cases}
  \Longrightarrow \quad
  R = 0.296 m
\end{equation*}

\begin{equation*}
  A R^{2/3} = 0.261
\end{equation*}

\begin{equation*}
  \dfrac{Q_{n}}{I^{1/2}} \cong A R^{2/3}
\end{equation*}

$\therefore ES\ CORRECTO$
