\underline{\bf Tramo 2}

\begin{equation*}
  r = 20\%
 \qquad
  h_{disp} = 1.83 m \,\text{(en progresiva }4.1 km \text{)}
 \qquad
  I = 0.00105
\end{equation*}

\begin{equation*}
  Q = 6.6 m^3/s
 \qquad
  V_{max} = 1.5 m/s
 \qquad
  n = 0.035
 \qquad
  m = 1
\end{equation*}

\begin{equation*}
  \text{Relación de huecos: 0.3}
  \qquad
  \text{Agua: poco limosa, limo muy fino}
\end{equation*}

Calculamos el área estable

\begin{equation*}
 A_{e} = \dfrac{Q}{V_{max}} = \dfrac{6.6 m^3/s}{1.5 m/s} = 4.4 m^2
\end{equation*}

Luego, aplicando la ecuación de Chazy Manning
\begin{equation*}
 V_{max} =  \frac{1}{n} R^{2/3} I^{1/2}
 \Longrightarrow \quad
 R = 2.06 m
\end{equation*}

\begin{equation*}
 P = \dfrac{A_{e}}{R} = \dfrac{4.4 m^2}{2.06 m} = 2.14 m
\end{equation*}

\begin{equation*}
  \begin{cases}
    A_e = B_{f}y + m y^{2} \\
    P = B_{f} + 2 y \sqrt{1 + m^{2}}
  \end{cases}
  \Longrightarrow \quad
  \text{sólo soluciones complejas}
\end{equation*}
