\underline{\bf Tramo 2}

\begin{equation*}
  r = 20\%
 \qquad
  h_{disp} = 1.83 m \,\text{(en progresiva }4.1 km \text{)}
 \qquad
  I = 0.00105
\end{equation*}

\begin{equation*}
  Q = 6.6 m^3/s
 \qquad
  V_{max} = 1.5 m/s
 \qquad
  n = 0.035
 \qquad
  m = 1
\end{equation*}

\begin{equation*}
  \text{Relación de huecos: 0.3}
  \qquad
  \text{Agua: poco limosa, limo muy fino}
\end{equation*}

Suelo cohesivo:
\begin{align*}
 \tau_{resistente\,talud} &= \tau_{resistente\,fondo} \\
 \tau_{act\,talud} = 0.75 \, \gamma \, y \, I &\leq \tau_{resistente\,talud} \\
 \tau_{act\,fondo} = 0.97 \, \gamma \, y \, I &\leq \tau_{resistente\,fondo} \Longrightarrow \text{más condicionante}
\end{align*}

Con la relación de huecos y el tipo de suelo, buscamos en la tabla que vincula fuerza tractiva con relación de huecos y obtenemos
el $\tau_{resistente\,fondo} = 0.38 lb/ft^2 = 1.855 kg/m^2$

\begin{align*}
 y &\leq \dfrac{\tau_{resistente\,fondo}}{0.97 \, \gamma \, I} \\
 y &\leq 1.82 m
\end{align*}

A pesar de que la altura máxima disponible es de $1.83 m$, las características del suelo nos limitan la $h_{disp}$ a $1.82 m$.
Considerando la revancha obtenemos $1.2 y = 1.82 m$, luego $y = 1.52 m$.

\begin{equation*}
  \begin{cases}
    Q = V \, A \\
    V =  \frac{1}{n} R^{2/3} I^{1/2}
  \end{cases}
  \Longrightarrow \quad
  Q = \frac{1}{n} R^{2/3} I^{1/2} A
\end{equation*}


\begin{equation*}
  \begin{cases}
    Q = \frac{1}{n} R^{2/3} I^{1/2} A \\
    R = \dfrac{A}{P} \\
    A = B_{f}y + m y^{2} \\
    P = B_{f} + 2 y \sqrt{1 + m^{2}}
  \end{cases}
  \Longrightarrow \quad
  \begin{cases}
    Q = \frac{1}{0.035} R^{2/3} 0.00105^{1/2} A \\
    R = \dfrac{A}{P} \\
    A = B_{f}1.52 m + (1.52 m)^2 \\
    P = B_{f} + 2 \, 1.52 m \sqrt{2}
  \end{cases}
  \Longrightarrow \quad
  B_{f} = 3.289 m
\end{equation*}

Ya que no se debe disminuir la sección aguas abajo, adoptamos $B_{f} = 4 m$, de modo que sea igual al utilizado en el tramo anterior.
Luego, calculamos el tirante correspondiente para el $B_{f}$ adoptado.

\begin{equation*}
  \begin{cases}
    Q = \frac{1}{n} R^{2/3} I^{1/2} A \\
    R = \dfrac{A}{P} \\
    A = B_{f}y + m y^{2} \\
    P = B_{f} + 2 y \sqrt{1 + m^{2}}
  \end{cases}
  \Longrightarrow \quad
  \begin{cases}
    Q = \frac{1}{0.035} R^{2/3} 0.00105^{1/2} A \\
    R = \dfrac{A}{P} \\
    A = 4 m y + y^2 \\
    P = 4 m + 2 y \sqrt{2}
  \end{cases}
  \Longrightarrow \quad
  y = 1.38 m
\end{equation*}

Luego,
\begin{equation*}
  \begin{cases}
    A = B_{f}y + m y^{2} \\
    P = B_{f} + 2 y \sqrt{1 + m^{2}}
  \end{cases}
  \Longrightarrow
  \begin{cases}
    A = 7.42 m^2 \\
    P = 7.90 m
  \end{cases}
  \Longrightarrow \quad
  R = 0.94 m
\end{equation*}


\underline{Verificación}

\begin{itemize}
 \item Tensiones
    \begin{align*}
    0.97 \, \gamma \, R \, S &\leq \tau_{resistente\,fondo} \\
    0.97 \, 1000 kg/m^3 \, 0.94 m \, 0.00105 &\leq \tau_{resistente\,fondo} \\
    0.957 kg/m^2 &\leq 1.855 kg/m^2
    \qquad
    \therefore VERIFICA 
    \end{align*}

  \item Escurrimiento
    \begin{itemize}
    \item De la ecuación de continuidad obtenemos la velocidad media:
	\begin{equation*}
	V_{m} = \dfrac{Q}{A} = \dfrac{6.6 m^3/s}{7.42 m^2} = 0.89 m/s
	\end{equation*}
    \item Para obtener la $V_{min}$, sabiendo el tirante y el tipo de agua, poco limoso de limo fino, buscamos en
    la tabla de velocidades mínimas y obtenemos $V_{min} = 0.53 m/s$.
	\begin{equation*}
	  \begin{cases}
	  V_{min} = 0.53 m/s \\
	  V_{max} = 1.5 m/s
	  \end{cases}
	  \Longrightarrow \quad
	  V_{m} = 0.89 m/s
	  \qquad
	  \therefore VERIFICA
	\end{equation*}
    \end{itemize}

 \item Desborde
    \begin{equation*}
    r = 20\%
    \qquad
    h_{disp} = 1.82 m
    \end{equation*}
    \begin{align*}
    y + r &\leq h_{disp} \\
    1.38 m + 0.2 \times 1.38 m &\leq h_{disp} \\
    1.66 m &\leq 1.82 m
    \qquad
    \therefore VERIFICA
    \end{align*}
\end{itemize}

